\section*{Abstract}
%Re-Os can be used as a cosmic clock from radioactive decay
The nuclide \re{187} decays to \os{187} with a halflife of $\simeq41.6 Gyr$, this makes it a candidate as a cosmic clock.

%examine uncertainties by repeating calculations of chemical evolution models, using random numbers drawn from a gaussian distribution to represent uncertianty
A hydrodynamical simulation is approximated with a simple one-zone chemical evolution model.
The physical parameters of the chemical evolution model is varied by generating random numbers with a gaussian distribution to emulate the uncertainty of parameters.

%Conclude that uncertainties in observed r-process abundances do not contribute much to the Re-Os cosmochronology and that Galactic history and uncertainties in observations have greater impact.
I conclude that observational constraints on r-process nuclei does not affect the cosmochronology of the Re-Os system much.
The Galactic history, like star formation and mass function over time, would have greater effect on the cosmochronology.
\newpage
