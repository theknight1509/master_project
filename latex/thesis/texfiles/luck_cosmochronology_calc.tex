
\renewcommand{\d}[1]{\frac{\textrm{d}#1}{\textrm{dt}}}
\newcommand{\pd}[2]{\frac{\partial #1}{\partial #2}}

Following the analytical approach of \mycite{luck80} and \mycite{shizuma05}, which follows the approach of \mycite{clayton64} to the chemical evolution of the \re{187}-\os{187}-system. \\
The solar values and evolution of \re{187}, \os{187}, \os{187} come, predominantly from three main sources;
s-process contribution of \os{186} and \os{187}, r-process contribution of \re{187},
and \betadecay from \re{187} to \os{187}.
A simple exponential form is adopted for the r-process contribution to \re{187}.

\begin{align*}
  \os{186}^\odot &= \os{186}^s \\
  \os{187}^\odot &= \os{187}^s + \os{187}^\beta \\
  \d{\os{187}^\beta} &= \lambda_\beta\re{187} \\
  \d{\re{187}} &= A(t) - \lambda_\beta\re{187} \\
  &= A_0e^{-\tau^{-1}t} - \lambda_\beta\re{187} \\
\end{align*}

\underline{Solving for \re{187}:}
\begin{align*}
  \d{\re{187}} + \lambda_\beta \re{187} &= A_0e^{-\tau^{-1}t} \\
  \textrm{General solution to homogenous equation:}
  & \hspace{3em} \dot{\re{187}}_h + \lambda_\beta \re{187}_h = 0 \\
  & \hspace{3em} \re{187}_h = Re(0) e^{-\lambda_\beta t} \\
  \textrm{Particular solution:}
  & \hspace{3em} \dot{\re{187}}_p = A_1 e^{-\tau^{-1}t} \\
  \re{187} = \re{187}_h + \re{187}_p &= Re(0) e^{-\lambda_\beta t} + A_1 e^{-\tau^{-1}t} \\
  \textrm{Initial condition of \re{187}:}
  & \hspace{3em} \re{187}(t=0) = 0 = Re(0) + A_1 \\
  & \hspace{3em} Re(0) = - A_1 \\
  \re{187} &= A_1 (e^{-\tau^{-1}t} - e^{-\lambda_\beta t}) \\
\end{align*}
Where $A_0$ and $A_1$ are scaling factors for the proposed form of r-process contribution, and $tau^{-1}$ is the
``decay constant'' of the proposed form of r-process contribution.
$\lambda_\beta=\frac{\ln 2}{T_{1/2}}$ is the decay constant of radioactive \re{187}, and $T_{1/2}$ is the half-life of radioactive \re{187}.

\underline{Solving for \os{187}:}
\begin{align*}
  \d{\os{187}^\beta} &= \lambda_\beta \re{187} = \lambda_\beta A_1 (e^{-\tau^{-1}t} - e^{-\lambda_\beta t}) \\
  \os{187}^\beta(\Delta t) &= \lambda_\beta A_1 \int_0^{\Delta t} (e^{-\tau^{-1}t} - e^{-\lambda_\beta t}) \textrm{dt} \\
  &= \lambda_\beta A_1 \left[ \frac{1}{\tau^{-1}} (1-e^{-\tau^{1}\Delta t}) - \frac{1}{\lambda_\beta} (1-e^{-\lambda_\beta \Delta t}) \right] \\
  &= A_1 \left[ \frac{\lambda_\beta}{\tau^{-1}} (1-e^{-\tau^{1}\Delta t}) - (1-e^{-\lambda_\beta \Delta t}) \right] \\
\end{align*}
Where the constants are the same as \re{187}.

\importantcomment{Rename $\frac{\os{187}}{\re{187}}$ to $f_{187}$}
\newcommand\fraction{\frac{\os{187}}{\re{187}}}
\newcommand\fracfsos{f_{187}(t_{sos})}
\newcommand\fracnow{f_{187}(t_{now})}
\underline{Calculating fraction of \os{187}/\re{187} ($f_{187}$ from here on):}
\begin{equation}
  \label{eq:luck-fractio}
  \frac{\os{187}^\beta(\Delta t)}{\re{187}(\Delta t)} \equiv f_{187} = \frac{\frac{\lambda_\beta}{\tau^{-1}} (1-e^{-\tau^{1}\Delta t}) - (1-e^{-\lambda_\beta \Delta t})}{e^{-\tau^{-1} \Delta t} - e^{-\lambda_\beta \Delta t}}
\end{equation}
Where $\Delta t$ is the time between the formation of the galaxy and the formation of the \sos. This is, according to the model, all the time available to produce r-process isotopes outside of the \sos before collapse.

Adopting meteoritic abundances for the solar values at the formation of the solar system from \mycite{shizuma05} (which adopts the values from \mycite{faestermann98}), and assuming the uncertainties of \re{187} and \os{187} are uncorrelated:
\begin{equation}
  \label{eq:shizuma-frac}
  \begin{array}{cc}
    \re{187}^\odot/\os{186}^\odot &= 3.51 \pm 0.09 (\pm 2.56\%) \\
    \os{187}^\odot/\os{186}^\odot &= 0.793 \pm 0.001 (\pm 0.126\%) \\
    \os{187}^\odot/\re{187}^\odot = f_{187} &= \frac{0.793}{3.51} \pm \sqrt{(2.56\%)^2 + (0.126\%)^2} \\
    &= 0.226 \pm 2.563\% (\pm 5.79\times 10^{-3})
  \end{array}
\end{equation}

Since \os{187} and \re{187} is separated from the galaxy after the time of formation of the \sos,
the fraction \os{187}/\re{187} at that time can be caluclated from simple \betadecay of \re{187}.
The time of solar system formation is denoted $t_{f,sos}$, and the current time is denoted $t_0$.
Note that the normalization and integration constants used from here on are not related to the constants used previously.

\begin{align*}
  \re{187}(t) &= A e^{-\lambda_\beta (t-t_{f,sos})} \\
  \d{\os{187}} &= -\d{\re{187}} = \lambda_\beta A e^{-\lambda_\beta (t-t_{f,sos})} \\
  \os{187} &= - A e^{-\lambda_\beta (t-t_{f,sos})} + C \\
  \textrm{\footnotesize normalization and integration constants:} &
  \hspace{3em}
  \begin{array}{rl}
    C &= \os{187}(t=t_{f,sos}) \\
    A &= \re{187}(t=t_{f,sos})
  \end{array} \\
  \fraction = f_{187}(t) &= \frac{C - Ae^{-\lambda_\beta (t-t_{f,sos})}}{Ae^{-\lambda_\beta (t-t_{f,sos}})}
  = \frac{C}{A}e^{\lambda_\beta (t-t_{f,sos})} - 1 \\
  f_{187}(t_0) &= \frac{C}{A}e^{\lambda_\beta (t_0-t_{f,sos})} - 1 \\
  & \Rightarrow \frac{C}{A} = \left[ f_{187}(t_0) + 1\right]e^{-\lambda_\beta (t_0-t_{f,sos})} \\
  f_{187}(t_{f,sos}) &= \frac{C}{A}e^{\lambda_\beta (t_0-t_{f,sos})} - 1
\end{align*}
This leads to an equation for calculating the Os-Re-fraction, $f_{187}(t_{f,sos})$, at the formation of the \sos from physical parameters.
The physical parameters are; the current Os-Re-fraction, $f_{187}(t_{0})$, the decay-rate of \re{187}, $\lambda_\beta$, and the age of the solar system, $t_0-t_{f,sos}$.
\begin{equation}
  \label{eq:frac-fsos}
  f_{187}(t_{f,sos}) = \left[ f_{187}(t_0) + 1\right]e^{-\lambda_\beta (t_0-t_{f,sos})} - 1
\end{equation}

\underline{Estimates of the physical parameter:}
\begin{equation}
  \label{eq:parameter-sources}
  \begin{array}{rl}
  \lambda_\beta &= \ln(2)/T_{1/2} \\
  T_{1/2} &= 41.577 \pm 0.12 Gyr \quad \textrm{from \mycite{snelling15}} \\
  t_0 - t_{f,sos} = \Delta t &= 4,568.2^{+0.2}_{-0.4} Myr
  = 4.568^{+0.2\times 10^{-3}}_{-0.4\times 10^{-3}} Gyr \quad \textrm{from \mycite{bouvier10}} \\
  \fracnow &= 0.226 \pm 0.0579 \quad \textrm{from eq.\ref{eq:shizuma-frac}}
  \end{array}
\end{equation}

\underline{The average value for $\fracfsos$ from eq. \ref{eq:frac-fsos}:}
\begin{equation}
  \fracfsos = \left[ \fracnow + 1 \right]e^{-\lambda_\beta \Delta t} - 1 = 0.136
\end{equation}

\underline{Error propagation of $\fracfsos$ from eq. \ref{eq:frac-fsos}:}
\begin{align*}
  \left( \frac{\delta f(x,y,z)}{f(x_0, y_0, z_0)} \right)^2 &=
  \left(\pd{f}{x}\right)_{(x_0,y_0,z_0)}^2 \left( \frac{\delta x}{x_0} \right)^2
  + \left(\pd{f}{y}\right)_{(x_0,y_0,z_0)}^2 \left( \frac{\delta y}{y_0} \right)^2
  + \left(\pd{f}{z}\right)_{(x_0,y_0,z_0)}^2 \left( \frac{\delta z}{z_0} \right)^2 \\
  \left(\frac{\delta \fracfsos}{\fracfsos}\right)^2 &= \left\{
  \begin{array}{l}
    \left(\pd{\fracfsos}{\Delta t}\right)_{(\Delta t, T_{1/2}, \fracnow)}^2 \left( \frac{\delta \Delta t}{\Delta t} \right)^2 \\
    + \left(\pd{\fracfsos}{T_{1/2}}\right)_{(\Delta t, T_{1/2}, \fracnow)}^2 \left( \frac{\delta T_{1/2}}{T_{1/2}} \right)^2 \\
    + \left(\pd{\fracfsos}{\fracnow}\right)_{(\Delta t, T_{1/2}, \fracnow)}^2 \left( \frac{\delta \fracnow}{\fracnow} \right)^2
  \end{array} \right. \\
  \pd{\fracfsos}{\Delta t} &= \pd{(eq.\ref{eq:frac-fsos})}{\Delta t}
  = (\fracnow + 1) \left(\frac{-\ln 2}{T_{1/2}}\right) e^{-\ln 2 \Delta t /T_{1/2}} \\
  \pd{\fracfsos}{T_{1/2}} &= \pd{(eq.\ref{eq:frac-fsos})}{T_{1/2}}
  = (\fracnow + 1) \left(\frac{\ln 2 \Delta t}{T_{1/2}}^2\right) e^{-\ln 2 \Delta t /T_{1/2}} \\
  \pd{\fracfsos}{\fracnow} &= \pd{(eq.\ref{eq:frac-fsos})}{\fracnow}
  = e^{-\ln 2 \Delta t /T_{1/2}} \\
  \left(\frac{\delta \fracfsos}{\fracfsos}\right)^2 &= \left\{
  \begin{array}{l}
    \left( (\fracnow + 1) \left(\frac{-\ln 2}{T_{1/2}}\right) e^{-\ln 2 \Delta t /T_{1/2}}
    \right)^2%_{(\Delta t, T_{1/2}, \fracnow)}
    \left( \frac{\delta \Delta t}{\Delta t} \right)^2 \\
    + \left((\fracnow + 1) \left(\frac{\ln 2 \Delta t}{T_{1/2}}^2\right) e^{-\ln 2 \Delta t /T_{1/2}}
    \right)^2%_{(\Delta t, T_{1/2}, \fracnow)}
    \left( \frac{\delta T_{1/2}}{T_{1/2}} \right)^2 \\
    + \left(e^{-\ln 2 \Delta t /T_{1/2}}\right)^2%_{(\Delta t, T_{1/2}, \fracnow)}
    \left( \frac{\delta \fracnow}{\fracnow} \right)^2
  \end{array} \right. \\
  &= e^{-2\ln 2 \Delta t/ T_{1/2}} \left[
    \begin{array}{l}
    \left( \frac{\delta \fracnow}{\fracnow} \right)^2 \\
    + \left((\fracnow + 1)\right)^2\left(\frac{ln 2}{T_{1/2}}\right)^2 \\
    \qquad \times \left[
      \left(\frac{\Delta t}{T_{1/2}}\right)^2 \left( \frac{\delta T_{1/2}}{T_{1/2}} \right)^2
      - \left( \frac{\delta \Delta t}{\Delta t} \right)^2
      \right]
    \end{array}
    \right]
  \intertext{
    Inserting values for ($\Delta t$, $T_{1/2}$, $\fracnow$) with uncertainties from eq.\ref{eq:parameter-sources}
    \newline\noindent
    $ \Delta t = 4.5682 Gyr \quad \delta \Delta t = 0.4 \times 10^{-3} Gyr$ \newline\noindent
    $T_{1/2} = 41.577 Gyr \quad \delta T_{1/2} = 0.12 Gyr $ \newline\noindent
    $ \fracnow = 0.226 \quad \delta \fracnow = 57.9 \times 10^{-3} $
  }
  &= 0.0564 \\
  \delta \fracfsos &= \sqrt{0.0564} \fracfsos = 0.0323
\end{align*}

\noindent\comment{Redo for upper \textit{and} lower uncertainty of $T_{1/2}$}

\begin{table}
  \begin{tabular}{|c|c|c|}
    \hline Parameter & Value & Source \\
    \hline
    \hline $T_{1/2}$ & $41.577 \pm 0.12$ [Gyr] & \mycite{snelling15} \\
    \hline $\Delta t$ & $4.5682 \pm 0.4 \times 10^{-3}$ [Gyr] & \mycite{bouvier10} \\
    \hline $\fracnow$ & $ 0.226 \pm 0.0579$ & eq.\ref{eq:shizuma-frac} (from \mycite{shizuma05}) \\
    \hline $\fracfsos$ & $0.136 \pm 0.0323$ & eq.\ref{eq:parameter-sources} \\
    \hline
  \end{tabular}
  \caption[Summary of parameters at \sos\ formation]{\label{tab:obs-cosmo-chronology}
    Adopted and calculated values for the halflife of \re{187}, age of \sos, cosmic clock fraction $f_{187} = \os{187}/\re{187}$ at current time, and at the time of formation of the \sos.
  }
\end{table}
