%use \figwidth for width of image
\begin{figure}
  \centering
  \includegraphics[width=\figwidth]{img/uncertainty_diagram_cote16a_fig1.png}
  \caption[\mycitetwo{cote16a}{fig.1}]{ \label{img:uncertainty-diagram}
    Qualitative visualization of how uncertainties accumulate in galactic chemical evolution models.
    Experimental data on nuclear reaction rates are uncertain to some degree. The change in rate and uncertainty in stellar conditions are not well known.
    The conditions inside a star of a given mass and metallicity come from 1 dimensional hydrodynamical simulations.
    The combined result from nuclear reactions and hydrodynamical simulations give a stellar models.
    The stellar models are applied to a simple stellar population to account for all the billions of stars in galaxy.
    These stellar models are then applied to a large scale hydrodynamical simulation (like \eris), or a semianalytical galaxy model (like \omegamodel).
    All the steps make assumptions and add uncertainty to the grand total uncertainty that is difficult to map in completeness.

    Diagram is taken from \mycitetwo{cote16a}{fig.1}
  }
\end{figure}
