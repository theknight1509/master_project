
After the big bang and the elements formed, the chart of nuclides was scarcsly filled.
Only the lightest elements and isotopes, hydrogen and helium primarily, filled the vast universe\cite{alphabetagamma}, in addition to the dark matter.
In the universe today we see much more heavier elements, and these must have been synthesised in some manner.
Nuclear fusion in stars create heavier elements up to iron, the heavier elements are made as by-products in which heavy atoms accumulates neutrons or protons and climb the chart of nuclides\cite{BBFH}.

Since \re{187} is shielded from the s-process, and \os{187} is shielded from the r-process, the synthesis of these isotopes are not
strongly coupled together. Granted, they both require heavy seed isotopes, hot neutron rich environments, and some method of ejection. The r-process needs so much higher neutron densities that it is believed to exist in vastly different stellar environments than the s-process\comment{(need some citation on the locations of r-process)}.
After \comment{insert halflife of re-187 here} half of \re{187} would have decayed to \os{187}, similar to the decay of \isotope{carbon}{6}{14} used in dating millenia old archelogical artifacts.
By taking the expected and observed abundance of the daughter and/or parent nuclei the age is determined by the ratio of isotopes and the halflife of the radioactive process.

By making a model of all the s-process and r-process sources in this Galaxy, the amount of synthesised \re{187} and \os{187} can be estimated. These estimates can be used to estimated the age of nucleosynthesis (when the heavy elements started forming).
Such a model is based on a mosaic of different scientific disciplines and data, and therefor also has a series of uncertainties from atomic properties to galactic history to numerical precision and everything in between. These uncertainties make the final image blurred.

This was attempted analytically by \comment{clayton + other articles} and they yielded \comment{value uncertainty from articles} and the results are useless compared to cosmological results \comment{add cosmological age of universe, galaxy etc} due to the huge uncertainties. However since the age of the Galaxy is known from cosmology, the biggest uncertainties in the synthesis-model can be constricted.

\section{Outline}

This thesis is divided into six chapters
\begin{description}
\item[Introduction]
\item[Theory I]
\item[Theory II]
\item[Methodology]
\item[Results]
\item[Conclusion and discussion]
\end{description}

\comment{Explain [X/Y]} \\
\comment{introduce delay-time somewhere} \\
\comment{pure DM vs. gas vs. postproduction} \\
\comment{astration on \re{187}} \\
