\newcommand{\imagefolder}{results/plots_fitting}

\section{Fitting of models}

In order to have the one-zone model \omegamodel\ best reproduce the \eris\ simulation \\
\comment{... continue introduction and description} \\
Some parameters are decidedly locked from the \eris\ simulation directly.
One of the most valuable result from \eris\ (for these purposes)
are the star formation rate thorugh Galactic time (also known as star formation history). The Galctic age in \eris\, is 14Gyr.
In order to produce stars, a mass function has to be set. A mass function is the statistical probability distribution of mass for a population of stars. In \eris\ the Kroupa94
(\comment{insert reference here})\\
(\comment{insert image of distribution here?})
mass function is used, and the same shall also be used for \omegamodel. 
The stellar synthesis in \eris\ postproduction comes from core collapse supernova, type 1a supernova and binary neutron star mergers.
In the appropriate \omegamodel\ the black hole - neutron star mergers shall not be taken into effect,
and the yield table for binary neutron star mergers is chosen to be \comment{insert reference to Arnould} \comment{add comment/description about how the yield table is the r-process from the sun}, because it contains \re{187}.

\comment{define new commands for MWOmega/MWCteOmega/fiducccial model}
\newcommand\mwomega{TEMP-MWOmega}
\newcommand\mwcomega{TEMP-MWcteOmega}
\newcommand\fiduccialomega{TEMP-fiduccial}
\importantcomment{introduce 'our' \omegamodel\ model as the concept \textit{'fiduccial model'}}

\FloatBarrier
\subsection{Inserting parameters directly}
\iffalse
Filenames:
set_sfr_plot_gas_mass.png                
set_sfr_plot_sfr.png                     
set_sfr_plot_spectro.png                 
set_sfr_plot_stellar_mass.png
\fi
\comment{describe parameters: sfr, tend, imf, BNSM/BHNSM, yield-tables etc.}
\comment{fix stellar mass plot data}
The first step towards finding an appropriate parameter space for \omegamodel\ is to make \omegamodel\ follow the stellar evolution of \eris. This is achieved by setting the initial mass function to Kroupa93\comment{(insert reference)}, and the star formation history from the \eris\ simulation. By activating type 1a supernovae and binary neutron star mergers, the stellar evolution of \omegamodel\ should be similar in nature to \eris. In the star formation history of \eris, the endtime is 14Gyr, and the endtime for \omegamodel should be set to the same value. There is only one(out of two) available yield tables for binary neutron star mergers that contain output for \re{187}, in the interest of this project we naturally choose this one\comment{(add reference to yield tables)}.

\begin{figure}[h]
  \centering
  \includegraphics[scale=0.5]{\imagefolder/set_sfr_plot_sfr.png}
  \caption{\label{img:fit-v0-sfr}
    Star formation rate(measured in solar masses of stars formed from gas each year) for four models of \omegamodel\ versus time. 
    \textit{'Eris'} refers to data directly from \eris -simulation. \textit{'Omega' default} refers to the \omegamodel\ model with no change to the initial parameters (see description in section \ref{sec:omega}). \textit{'Omega' MW} refers to the \omegamodel\ model with the Milky Way parameter (see description in section \ref{sec:omega}), and \textit{'Omega' MW cte} refers to the same but with a constant one-solar-mass-per-year star formation rate. \textit{'Omega' w/'Eris'-SFR} refers to the \omegamodel\ model with the star formation and mass function from \eris.
    Firstly it is clear that star-formation is suppressed for all models except \textit{'Omega' default}. This is from the lack of gas to create stars from. Secondly the \textit{'Omega' w/'Eris'-SFR} model is the only model to accurately reproduce the \eris\ star formation at early times. While both \textit{'Omega' MW} and \textit{'Omega' MW cte} are meant to represent the milky way, they cannot be used to accurately represent \eris.
  }
\end{figure}
\begin{figure}[h]
  \centering
  \includegraphics[scale=0.5]{\imagefolder/set_sfr_plot_stellar_mass.png}
  \caption{\label{img:fit-v0-stellarmass}
    Total accumulated stellar mass(cumulutaive sum of stellar mass produces from gas, measured in solar masses) for four models of \omegamodel\ versus time.
    \textit{'Eris'} refers to data directly from \eris -simulation. \textit{'Omega' default} refers to the \omegamodel\ model with no change to the initial parameters (see description in section \ref{sec:omega}). \textit{'Omega' MW} refers to the \omegamodel\ model with the Milky Way parameter (see description in section \ref{sec:omega}), and \textit{'Omega' MW cte} refers to the same but with a constant one-solar-mass-per-year star formation rate. \textit{'Omega' w/'Eris'-SFR} refers to the \omegamodel\ model with the star formation and mass function from \eris.
    This graph also shows that stellar production is suppressed at early time from lack of gas. Small amount of stars are still created from the enriched gas expelled by dying stars at late times, however this is a small contribution to the stellar production. Only \textit{'Omega' w/'Eris'-SFR} can accurately reproduce \textit{'Eris'} at early times, unlike the other \omegamodel\ models.
  }
\end{figure}
\begin{figure}[h]
  \centering
  \includegraphics[scale=0.5]{results/plots_fitting/set_sfr_plot_gas_mass.png}
  \caption{\label{img:fit-v0-gasmass}
    Mass of gas in the interstellar medium for four different models of \omegamodel, and the \eris\ simulation against time in Gyrs.
    \textit{'Eris'} refers to data directly from \eris -simulation. \textit{'Omega' default} refers to the \omegamodel\ model with no change to the initial parameters (see description in section \ref{sec:omega}). \textit{'Omega' MW} refers to the \omegamodel\ model with the Milky Way parameter (see description in section \ref{sec:omega}), and \textit{'Omega' MW cte} refers to the same but with a constant one-solar-mass-per-year star formation rate. \textit{'Omega' w/'Eris'-SFR} refers to the \omegamodel\ model with the star formation and mass function from \eris.
    The gas, that is the foundation for star formation, is used up before 2 Gyrs for all models except for \textit{'Omega' default}.
  }
\end{figure}
The main issue for all models is clear: star formation uses up all the gas in the model and star formation is quenched.

\comment{do I skip the spectroscopic plots?}
\iffalse %blockcomment
\begin{figure}
  \centering
  \includegraphics[scale=0.5]{\imagefolder/set_sfr_plot_spectro.png}
  \caption{\label{fig:fit-v0-spectro} \comment{Explain each legend thouroughly}}
\end{figure}
\fi %end blockcomment

\FloatBarrier
\subsection{Modifying masses}
\iffalse
Filenames:
set_mass_1_plot_stellar_mass.png
set_mass_1_plot_total_mass.png    
set_mass_2_plot_stellar_mass.png  
set_mass_2_plot_total_mass.png    
set_mass_3_plot_sfr.png           
set_mass_3_plot_spectro_iron.png  
set_mass_3_plot_spectro_oxy.png   
set_mass_3_plot_stellar_mass.png  
set_mass_3_plot_total_mass.png
\fi
\comment{what are realistic masses, outflows, inflows?}
\comment{explain next step of process}

In order to produce enough stars to reproduce \eris\ the galaxy-model must have more gas. The \omegamodel\ supports inflow of primordial gas from the medium around the galaxy, and outflow of chemically enriched gas into the surrounding medium. However, since \omegamodel is a \textit{one-zone} model, the chemically enriched material cannot return from the surrounding medium. That would require a two-zone model (or reater). A constant rate will be used for inflow, while a outflow rate proportional to the supernova rate will be used to create a more realistic model within the restrictions og \omegamodel.

%mass-table from Guedes10
\begin{table}
  \begin{tabular}{|c|c|c|c|c|}
    \hline
    $f_b$ [\,] & $z$[\,] & $M_{vir}$[$10^{11}\msol$] & $M_b$[$10^{10}\msol$] & $t$[Gyr] \\
    \hline
    0.121 & 0.0 & 7.9 & 9.6 & 13.724 \\
    0.126 & 1.0 & 5.4 & 6.8 & 6.075 \\
    \hline
  \end{tabular}
  \caption[Mass data \eris]{\label{tab:guedes11-baryonic-mass}
    From \comment{Guedes10 table 1}, $f_b$ is the baryonic mass fraction of the galaxy, $z$ is the redshift in the simulation, $M_{vir}$ is the virial mass of the halo, $M_b$ is the total baryonic mass within the halo(multiplication of $f_b$ and $M_{vir}$), $t$ is the time of the corresopnding redshift.
    Time is calculated from redshift using Ned Wright's cosmology calculator(February 12th 2018)\comment{reference to cosmology calculator article here} with the cosmological parameters, $H_0=73[km s^{-1} Mpc^{-1}]$, $\Omega_M=0.24$, and $\Omega_\Lambda=1-\Omega_M=0.76$ for a flat universe as stated in \mycite{guedes11e}.}
\end{table}

From table \ref{tab:guedes-baryonic-mass} the total baryonic content of the galaxy is known at redshift zero and one. This information is used to fix the initial mass of primordial gas and inflow of primordial gas. 


\begin{figure}[h]
  \centering
  \includegraphics[scale=0.5]{\imagefolder/set_mass_1_plot_total_mass.png}
  \caption{\label{fig:fit-v1-1-total}
    The total baryonic mass of the \omegamodel-model for four different initial/inflow parameters.
    $M_0$ is the initial primordial gas of the galaxy(in \msol), $\dot{M}$ is the inflow (in \msol/yr).
    This visualization shows that 44G\msol and 3.7\msol/yr are the optimal parameters to reproduce the two baryonic data-points from \eris, although more then these four were tried.
  }
\end{figure}
\begin{figure}[h]
  \centering
  \includegraphics[scale=0.5]{\imagefolder/set_mass_1_plot_stellar_mass.png}
  \caption{\label{fig:fit-v1-1-stellar}
    Plotting the cumulative stellar mass formed in the inflow-\omegamodel-models. All four reproduce the \eris\ cumulative star formation, because these models have enough gas to form the stars.
  }
\end{figure}

Supernova feedback will drive an outflow from the galaxy into the surrounding medium \comment{find appropriate reference}. Adding outflow proportional to the supernova rate adds some realisim to the model, and might reproduce some of the spectroscopic features.
In \omegamodel this is activated with the parameters \verb|mass_loading|(which ejects a amount of gas relative to the stellar mass formed), and \verb|out_follows_E_rate|(which adds a timedelay to the outflow, making the outflow proportional to the supernova rate instead of the star formation rate).
Outflow removes gas from the galaxy, or interstellar medium, lowering the total amount of mass in the galaxy. Therefor the initial primordial gas and constant inflow must be increased as well.

\begin{figure}[h]
  \centering
  \includegraphics[scale=0.5]{\imagefolder/set_mass_2_plot_total_mass.png}
  \caption{\label{fig:fit-v1-2-total}
    Total baryonic mass of galaxy over time.\comment{what is the initial mass of gas and inflow rate?}
    The outflow adds a non-linear effect to the total mass.
  }
\end{figure}
\importantcomment{add spectroscopic outflow plot here!}
\begin{figure}[h]
  \centering
  \includegraphics[scale=0.5]{\imagefolder/set_mass_2_plot_stellar_mass.png}
  \caption{\label{fig:fit-v1-2-stellar}
    Cumulative stellar mass formed against time for \comment{X} \omegamodel\ models, and the \eris-simulation.
    The outflow removes mass, but there is still enough gas to form stars from the \eris\ star formation rate.
  }
\end{figure}
\begin{figure}[h]
  \centering
  \includegraphics[scale=0.5]{\imagefolder/set_mass_2_plot_iron.png}
  \caption{\label{fig:fit-v1-2-iron}
    Iron abundance in the models with varying mass-loading parameters(solar masses of outflow per solar mass of supernova). The data from \eris\ show two 'dips' from the increasing tendency.
    The dips could not be reproduced by outflow of enriched material and inflow of hydrogen. Outflow peaks over the 'dips', reducing the spectroscopic abundance, however the effect is wide and smeared out over a time range beyond both 'dips'.
  }
\end{figure}

Setting the initial mass, inflow and outflow, gives the desired star formation. A final comparison of the fiducial \omegamodel model, the predefined models (\mwomega, \mwcomega, and \omegamodel with all default parameters), and the data from the \eris\ simulation.
For the predefined models, the initial mass of primordial gas have been increased to $9.6\times10^{10}\msol$(the final value baryon-mass from \eris) to show the full evolution of star formation.
Two prominent features in the spectroscopic data of \eris\ is the two 'dips' around universal time t=5 and t=8 Gyrs. These dips might be reproduced by adding primordial inflow, hydrogen, and enriched outflow, concentrated on on those periods($t\simeq5Gyr$ and $t\simeq8Gyr$), since these periods might coincide with the death of stars from the star forming peak in figure \ref{img:fit-v0-sfr}.
Varying supernova-related outflow(known as the \texttt{mass\_loading} parameter) gives an expected result. In figure \ref{fig:fit-v1-2-iron} variation in the spectroscopic iron abundance can be seen the desired region, around the two 'dips', but the effect is too small to reproduce the two dips. The effect is also too wide and more closely similar to one big 'dip'. One unexpected result is that the smalles \texttt{mass\_loading} parameter yields the lowest dip(not really a dip at all, but more flat in the desired direction). suggesting that the outflow from supernovae occur later than the two 'dips'.
This means that the two 'dips' cannot be reproduced by outflow and inflow.
\comment{what are mass parameters now?}

\begin{figure}[h]
  \centering
  \includegraphics[scale=0.5]{\imagefolder/set_mass_3_plot_total_mass.png}
  \caption{\label{fig:fit-v1-3-total}
    Total baryonic mass of galaxy over time for \eris, \fiduccialomega, \mwomega, \mwcomega\ and \omegamodel\ with all default parameters.
    Only the \fiduccialomega model reproduces the total mass content found in \eris, represent by two datapoints from table \ref{tab:guedes-baryonic-mass}.
  }
\end{figure}
\begin{figure}[h]
  \centering
  \includegraphics[scale=0.5]{\imagefolder/set_mass_3_plot_stellar_mass.png}
  \caption{\label{fig:fit-v1-3-stellar}
    Cumulative stellar mass formed over time for \eris, \mwomega, \mwcomega, \fiduccialomega\ and \omegamodel\ with all default parameters.
    All predefined models massively undershoots or overshoots the measured star formation in \eris.
    The \fiduccialomega\ model accurately reproduces the cumulative stellar formation with \eris. The slight variation between the \fiduccialomega\ model and \eris\ is due to low numerical resolution.
  }
\end{figure}
\begin{figure}[h]
  \centering
  \includegraphics[scale=0.5]{\imagefolder/set_mass_3_plot_spectro_iron.png}
  \caption{\label{fig:fit-v1-3-iron}
    Spectroscopic iron over time for \eris, \mwomega, \mwcomega, \fiduccialomega\ and \omegamodel\ with all default parameters.
    The \fiduccialomega\ model has almost no star formation in the very beginning of the integration, this leads to a delayed chemical evolution that can be seen in the graph. The predefined models have some(if not much) star formation from the first integration step to the last. This implies that chemical evolution can begin much sooner, as can be seen in the graphs.
  }
\end{figure}
\begin{figure}[h]
  \centering
  \includegraphics[scale=0.5]{\imagefolder/set_mass_3_plot_spectro_oxy.png}
  \caption{\label{fig:fit-v1-3-oxy}
    Spectroscopic oxygen over time for \eris, \mwomega, \mwcomega, \fiduccialomega and \omegamodel with all default parameters.
    The \fiduccialomega\ model has almost no star formation in the very beginning of the integration, this leads to a delayed chemical evolution that can be seen in the graph. The predefined models have some(if not much) star formation from the first integration step to the last. This implies that chemical evolution can begin much sooner, as can be seen in the graphs.
  }
\end{figure}

\FloatBarrier
\subsection{Effect of AGB stars, massive stars, population III stars and Type 1a Supernovae}
\iffalse
set_star_plot_pop3_bound_iron.png
set_star_plot_pop3_bound_oxy.png
set_star_plot_pop3_yt_iron.png
set_star_plot_pop3_yt_oxy.png
set_star_plot_transmass.png
\fi
\iffalse
set_sn1a_plot_sn1a_dtd1_iron.png         
set_sn1a_plot_sn1a_dtd1_oxy.png          
set_sn1a_plot_sn1a_dtd2_iron.png         
set_sn1a_plot_sn1a_dtd2_oxy.png
set_sn1a_plot_sn1a_dtd3_iron.png
set_sn1a_plot_sn1a_dtd3_oxy.png
set_sn1a_plot_sn1a_num1_iron.png
set_sn1a_plot_sn1a_num1_oxy.png
set_sn1a_plot_sn1a_num2_iron.png
set_sn1a_plot_sn1a_num2_oxy.png
set_sn1a_plot_sn1a_yt.png
\fi

\comment{what parameters are used to mess with stars?}

Chemical enrichment of galactic gas (the interstellar medium), comes from stars.
\comment{quick recap of theory section} Hydrogen and helium from the primordial gas is locked into a star, where fusion processes transmutates the elements into heavier elements up to iron. In the process some heavier elements are created, mostly by neutron capture processes. At the end of the stars life some of the material will be ejected back into the interstellar medium.
Asymptotic giant branch stars are low mass stars at the end of their life, they eject mass via episodes known as helium flashes, leaving a white dwarf behind.
Massive stars end their life as typeII supernovae, ejecting most of their enriched material leaving a neutron star or black hole behind.
The very first stars, with no initial chemical enrichment, or metallicity, are called population III stars. They are generally believed to have a slightly different initial mass distribution function, and could produce slightly different distributions of metals.
The exact science of population III stars is not well defined, as none has been observed, but the stellar population is one of the options of the \omegamodel\ model and should therefor be taken into consideration when comparing \eris and \omegamodel.
The remnants, white dwarves, neutron stars and black holes, are not the end of the story, binary star systems can bring new life to these dead bodies. A white dwarf accreting plasma from the envelope of a binary star can accumulate enough mass to ignite a core-collapse that ejects more enriched matter into the interstellar medium. Two neutron stars in orbit around eachother can loose gravitational energy to gravitational waves and merge. Such an energetic event will create alot of heavy elements and eject alot of the mass of the binary system with great velocity. Similar gravitational events can occur between two black holes and a black hole and a neutron star. The last two event will be ignored because two black holes do not create or eject any heavy elements (or any elements at all), while black hole neutron star merger is not included in \eris.

\comment{add plot about agb/massive yield tables}
\begin{figure}[h]
  \centering
  \importantcomment{plot yield tables here}
  %\includegraphics[scale=0.5]{\imagefolder/set_star_plot_yt.png}
  %\caption{\label{fig:fit-v2-agbm-yt}}
\end{figure}
\begin{figure}[h]
  \centering
  \includegraphics[scale=0.5]{\imagefolder/set_star_plot_transmass.png}
  \caption{\label{fig:fit-v2-agbm-transmass}
    The transitionmass is the value where the star goes from being considered an asymptotic giant branch star to a massive star and usually considered to be 8\msol.
    Stars with initial mass below this threshold leave the main sequence to become asymptotic giant branch star that ejects enriched mass in helium flashes and leaves a white dwarf.
    Stars with initial mass above this threshold leave the main sequence, goes through the giant branch burning heavier layers of stellar material, ending their life as a core collapse supernova.
    It is clear from the plot that varying the transitionmass between 7\msol and 10\msol does not significantly change the yield output of the \omegamodel\ model
  }
\end{figure}
\begin{figure}[h]
  \centering
  \includegraphics[scale=0.5]{\imagefolder/set_star_plot_pop3_yt_iron.png}
  \caption{\label{fig:fit-v2-pop3-yt-iron}
    The plot shows iron abundance for \eris\ data and \omegamodel with different yield tables for population III stars.
    Population III stars are stars with no initial metallicity, meaning the first stars. These stars are believed to be bigger, but have not been observed.
    It is clear that the different yield tables gives no variation in iron abundance, even in early times.
  }
\end{figure}
\begin{figure}[h]
  \centering
  \includegraphics[scale=0.5]{\imagefolder/set_star_plot_pop3_yt_oxy.png}
  \caption{\label{fig:fit-v2-pop3-yt-oxy}
    The plot shows oxygen abundance for \eris\ data and \omegamodel with different yield tables for population III stars.
    Population III stars are stars with no initial metallicity, meaning the first stars. These stars are believed to be bigger, but have not been observed. The yield tables gives the isotopic ejecta from supernovae.
    It is clear that the different yield tables gives no variation in iron abundance, even in early times.
  }
\end{figure}
\begin{figure}[h]
  \centering
  \includegraphics[scale=0.5]{\imagefolder/set_sn1a_plot_sn1a_yt.png}
  %\caption{\label{fig:fit-v2-sn1a-yt}}
\end{figure}
\begin{figure}[h]
  \centering
  \includegraphics[scale=0.5]{\imagefolder/set_star_plot_pop3_bound_iron.png}
  \caption{\label{fig:fit-v2-pop3-imfb-iron}
    The plot shows iron abundance for \eris\ data and \omegamodel with different mass-function boundaries for population III stars.
    Population III stars are stars with no initial metallicity, meaning the first stars. These stars are believed to be bigger, but have not been observed. The boundaries of the mass function change the distribution of initial mass of the population III stars.
    It is clear that the different yield tables gives no variation in iron abundance, even in early times.
  }
\end{figure}
\begin{figure}[h]
  \centering
  \includegraphics[scale=0.5]{\imagefolder/set_star_plot_pop3_bound_oxy.png}
  \caption{\label{fig:fit-v2-pop3-imfb-oxy}
    The plot shows oxygen abundance for \eris\ data and \omegamodel with different mass-function boundaries for population III stars.
    Population III stars are stars with no initial metallicity, meaning the first stars. These stars are believed to be bigger, but have not been observed. The boundaries of the mass function change the distribution of initial mass of the population III stars.
    It is clear that the different yield tables gives no variation in iron abundance, even in early times.
  }
\end{figure}
\begin{figure}[h]
  \centering
  \includegraphics[scale=0.5]{\imagefolder/set_sn1a_plot_sn1a_num1_iron.png}
  %\caption{\label{fig:fit-v2}}
\end{figure}
\begin{figure}[h]
  \centering
  \includegraphics[scale=0.5]{\imagefolder/set_sn1a_plot_sn1a_num1_oxy.png}
  %\caption{\label{fig:fit-v2}}
\end{figure}
\begin{figure}[h]
  \centering
  \includegraphics[scale=0.5]{\imagefolder/set_sn1a_plot_sn1a_dtd1_iron.png}
  \caption{\label{fig:fit-v2-dtd1-iron}}
\end{figure}
\begin{figure}[h]
  \centering
  \includegraphics[scale=0.5]{\imagefolder/set_sn1a_plot_sn1a_dtd1_oxy.png}
  %\caption{\label{fig:fit-v2}}
\end{figure}
\begin{figure}[h]
  \centering
  \includegraphics[scale=0.5]{\imagefolder/set_sn1a_plot_sn1a_dtd2_iron.png}
  %\caption{\label{fig:fit-v2}}
\end{figure}
\begin{figure}[h]
  \centering
  \includegraphics[scale=0.5]{\imagefolder/set_sn1a_plot_sn1a_dtd2_oxy.png}
  %\caption{\label{fig:fit-v2}}
\end{figure}
\begin{figure}[h]
  \centering
  \includegraphics[scale=0.5]{\imagefolder/set_sn1a_plot_sn1a_dtd3_iron.png}
  %\caption{\label{fig:fit-v2}}
\end{figure}
\begin{figure}[h]
  \centering
  \includegraphics[scale=0.5]{\imagefolder/set_sn1a_plot_sn1a_dtd3_oxy.png}
  %\caption{\label{fig:fit-v2}}
\end{figure}
\begin{figure}[h]
  \centering
  \includegraphics[scale=0.5]{\imagefolder/set_sn1a_plot_sn1a_num2_iron.png}
  %\caption{\label{fig:fit-v2}}
\end{figure}
\begin{figure}[h]
  \centering
  \includegraphics[scale=0.5]{\imagefolder/set_sn1a_plot_sn1a_num2_oxy.png}
  %\caption{\label{fig:fit-v2}}
\end{figure}

\FloatBarrier
\subsection{Binary neutron star mergers}
\iffalse
Filenames:
set_nsm_plot_combo_rates.png      
set_nsm_plot_combo_spectro.png    
set_nsm_plot_dtd.png              
set_nsm_plot_ejmass.png           
set_nsm_plot_final_rates.png             
set_nsm_plot_final_spectro.png           
set_nsm_plot_mergerfraction_rates.png    
set_nsm_plot_mergerfraction_spectro.png  
set_nsm_plot_nbnsm_rates.png             
set_nsm_plot_nbnsm_spectro.png
\fi

\comment{what are realistic parameters? uncertainty of them?}
\comment{what are the input parameter-space used?}

\begin{figure}[h]
  \centering
  \includegraphics[scale=0.5]{\imagefolder/set_nsm_plot_dtd.png}
  \caption{\label{fig:fit-v3-dtd}
    Abundance of europium in \eris\ simulation and \omegamodel\ models for galactic time in Gyrs.
    There are two main ways to calculate the delay-time of a neutron star merger in \omegamodel: one is a powerlaw distribution in time(with boundaries at minimum and maximum time), while another is setting a time after which all neutron star binaries merge, called a coalescence time.
    \\ \comment{what does \eris\ use?} \\
    In order to reproduce the \eris\ spectroscopic abundances, \omegamodel\ must synthesize more europium at an earlier time, this is achieved by a steep distribution with early minimum-time. It is clear from the plot that all models behave similar at late times, regardless of delay-time distribution. There is also little difference between a short coalescence time or a powerlaw distribution with short minimum time.
  }
\end{figure}
\begin{figure}[h]
  \centering
  \includegraphics[scale=0.5]{\imagefolder/set_nsm_plot_ejmass.png}
  \caption{\label{fig:fit-v3-ejecta}
    Spectroscopic europium abundance against galactic time for \eris-data and several \omegamodel models. In the models the mass ejected from each neutron star merger have been modified.
    Modifying the mass ejected from each event will just scale the total europium content up and down.
    Ejecting 0.2-0.3 \msol per event gives a pretty decent fit to late time europium and early time europium.
    However for the 'dips' between 2 and 8 Gyrs, the \omegamodel\ model overshoots the \eris\ data.
  }
\end{figure}
\begin{figure}[h]
  \centering
  \includegraphics[scale=0.5]{\imagefolder/set_nsm_plot_mergerfraction_rates.png}
  \caption{\label{fig:fit-v3-mergerfrac-nsmr}}
\end{figure}
\begin{figure}[h]
  \centering
  \includegraphics[scale=0.5]{\imagefolder/set_nsm_plot_mergerfraction_spectro.png}
  \caption{\label{fig:fit-v3-mergerfrac-euro}}
\end{figure}
\begin{figure}[h]
  \centering
  \includegraphics[scale=0.5]{\imagefolder/set_nsm_plot_nbnsm_rates.png}
  \caption{\label{fig:fit-v3-number-nsmr}}
\end{figure}
\begin{figure}[h]
  \centering
  \includegraphics[scale=0.5]{\imagefolder/set_nsm_plot_nbnsm_spectro.png}
  \caption{\label{fig:fit-v3-number-euro}}
\end{figure}
\begin{figure}[h]
  \centering
  \includegraphics[scale=0.5]{\imagefolder/set_nsm_plot_combo_rates.png}
  \caption{\label{fig:fit-v3-combo-nsmr}}
\end{figure}
\begin{figure}[h]
  \centering
  \includegraphics[scale=0.5]{\imagefolder/set_nsm_plot_combo_spectro.png}
  \caption{\label{fig:fit-v3-combo-euro}}
\end{figure}
\begin{figure}[h]
  \centering
  \includegraphics[scale=0.5]{\imagefolder/set_nsm_plot_final_rates.png}
  \caption{\label{fig:fit-v3-nsmr}}
\end{figure}
\begin{figure}[h]
  \centering
  \includegraphics[scale=0.5]{\imagefolder/set_nsm_plot_final_spectro.png}
  \caption{\label{fig:fit-v3-final-euro}}
\end{figure}

\FloatBarrier
\subsection{Size of timesteps}
In the fitting process the timestep was chosen somewhat arbitrarily in order to give the best result in a realistic amount of time for working purposes. The size of the timesteps are important for accuracy of the result of a numerical calculation, longer and fewer timesteps give unaccurate results, but too short and too many timesteps take much memory and comupting time. numerical presicion and floating point arithmetic can also come into play at the shorter timestep simulations.

In order to check the accuracy of \omegamodel\ against \eris\ a comparison is needed.
\omegamodel\ uses two different timestepping tecnhiques, one with n logarithmic timesteps between start and end, and one with constant timesteps between start and end.
By varying the number of timesteps in the \omegamodel\ data for [O/H], [F/H], [Eu/H] in time will be created.
These data will be interpolated onto the time-data for \eris\ and compared to the \eris-data with a pearson chi-squared test. The pearson static is only used to compare the difference with increasing time-resolution, the chi-squared statistic will not be used to compare what is, in essence, two different simulations.

It should be noted that there were some peculiar difficulties discovered at this point. \omegamodel\ cannot create a stable list of timestep-values when the timesteps are shorter then the time-array in the input star formation rate (which, in this case belongs to \eris). The same problem also occured for some timestep-sizes close to the timestepsize of \eris.
This is the reason for the sparse datapoints at higher resolutions.

\begin{figure}
  \centering
  \begin{subfigure}
    \centering
    \includegraphics[width=0.5\textwidth]{results/resolution_analysis/resolution_difference_euro_logged.png}
    \caption{\label{fig:fit-res-euro}}
  \end{subfigure}
  \begin{subfigure}
    \centering
    \includegraphics[width=0.5\textwidth]{results/resolution_analysis/resolution_difference_euro_logged.png}
    \caption{\label{fig:fit-res-euro}}
  \end{subfigure}
  \begin{subfigure}
    \centering
    \includegraphics[width=0.5\textwidth]{results/resolution_analysis/resolution_difference_euro_logged.png}
    \caption{\label{fig:fit-res-euro}}
  \end{subfigure}
\end{figure}

\FloatBarrier
\subsection{Final parameters of fitting}
\comment{add plots from final bestfit-folder}

