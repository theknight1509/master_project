Conclusions:

\begin{itemize}
\item The nuclear uncertainties lead to large deviations in total amount of \re{187} and \os{187} in the interstellar medium (which finally collapses to the solar system. The corresponing uncertainty in \os{187}/\re{187} is therefor also large (see figures \ref{fig:MCexperiment-nodecay} in section \ref{sec:results-nodecay}).
\item Adding \betadecay leads to a dominating term in \os{187}/\re{187}. Since postprocessing does not include the uncertainty of \betadecay (the astration effect on \re{187}), this significantly lowers the uncertainty of \os{187}/\re{187}. See figures \ref{fig:MCExperiment-decay} in section \ref{sec:results-decay}.
\item The large uncertainties come from outliers, so removing them in a physically motivated manner gives far better results for the distribution in figures \ref{fig:MCExperiment-delmax} in section \ref{sec:results-delmax}.
\item The rate of supernovae and neutron star mergers follow the star formation rate, with a delay-time. In this model (by fitting \omegamodel\ to the data from \eris) a total of 100'000 neutron star mergers in the Galaxy were necessary to create the r-process metals observed in our \sos.
\item The data from this model coincides with a cosmochronology method between sudden and uniform, but not with Shizumas attempt. By using scipy to fit the analytical model to our results, a decent curve is found and the parameters have some significance \comment{need to add the parameters to table! What is $\leff$ compared to $\lambda_\beta$? how does $\lambda$ compare to NSM-rate?}. See figure \ref{fig:osre-model-fitting}.
\item Adding some uncertainty to the slope of initial mass funciton gives a non-gaussian term to the abundance of \re{187} and \os{187} (see figures \ref{fig:MCExperiment-imfslope-re187} and \ref{fig:MCExperiment-imfslope-os187}). This effect is greatly enhanced in \os{187}/\re{187} (figure \ref{fig:MCExperiment-imfslope-div}) and can be seen as an extra gaussian term in the distribution. The slope of the initial mass function is considered because \mycite{cote16} finds it to be the most important parameter in chemical evolution, due to it's control over the relative number of high-mass stars.
\item \comment{I am running a simulation with 10\% uncertainty on number of NSMs, their ejecta mass and the delay-time distribution. Just for compariosn and show}
  \item \comment{If I have time, I should probably find a way to vary the half-life of \re{187} and add some plots for that}
  \item The nuclear uncertainties of reaction rates does not affect the overall uncertainty much, astration effects on \re{187} as well as the history and galactic parameters determine the evoltuion even more than nuclear reaction rates.
\end{itemize}

Potential errors and discussion of conclusions:
\begin{itemize}
\item The removal of outliers is physically motivated and well justified.
\item The evolution of \os{187}/\re{187} is dominated by \betadecay, which can be seen by comparing figures \ref{fig:MCExperiment-nodecay} and \ref{fig:MCExperiment-decay} (with and without \betadecay). Since the \betadecay-implementation in postprocessing does not include the uncertainty of the \re{187} halflife (at least in those figures) alot of the real nuclear uncertainty \textit{vanishes}, this implemtation could be improved upon.
\end{itemize}

