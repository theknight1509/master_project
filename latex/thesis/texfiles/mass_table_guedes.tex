\begin{table}
  \begin{tabular}{|c|c|c|c|c|}
    \hline
    $f_b$ [\,] & $z$[\,] & $M_{vir}$[$10^{11}\msol$] & $M_b$[$10^{10}\msol$] & $t$[Gyr] \\
    \hline
    0.121 & 0.0 & 7.9 & 9.6 & 13.724 \\
    0.126 & 1.0 & 5.4 & 6.8 & 6.075 \\
    \hline
  \end{tabular}
  \caption[Mass data \eris]{\label{tab:guedes11-baryonic-mass}
    From \comment{Guedes10 table 1}, $f_b$ is the baryonic mass fraction of the galaxy, $z$ is the redshift in the simulation, $M_{vir}$ is the virial mass of the halo, $M_b$ is the total baryonic mass within the halo(multiplication of $f_b$ and $M_{vir}$), $t$ is the time of the corresopnding redshift.
    Time is calculated from redshift using Ned Wright's cosmology calculator(February 12th 2018)\comment{reference to cosmology calculator article here} with the cosmological parameters, $H_0=73[km s^{-1} Mpc^{-1}]$, $\Omega_M=0.24$, and $\Omega_\Lambda=1-\Omega_M=0.76$ for a flat universe as stated in \mycite{guedes11e}.}
\end{table}
