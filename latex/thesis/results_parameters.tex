\section{Uncertainty of parameters}

\subsection{Method}
For varying different parameters simultaneously a similar method was used as in section \ref{sec:results-yields}, where a ``fudge-factor'' was applied to the \eris\ best fitted parameter-values. The ``fudge-factor'' is distributed gaussian around 1.0 and the factor is the multiplied to the parameter-value.

%% The parameter values chosen are the ones that deal with r-process production; fraction of neutron stars that merge (number of mergers), the ejecta mass from a single neutron star merger, the delay-time distribution of merging events.
%% Also the relevant r-process isotopes \re{187}, \re{185}, \w{184}, \os{188}, and the relevant s-process isotopes \os{187}, \os{186}.
%% These are chosen because they lie close to the \re{187}-\os{187} pair in the chart of nuclides and also closely related to the s-process path (and branching path).

\subsubsection{Monte Carlo experiment}
%add code snippet and explain how omega was used after fitting
%explain data
\subsubsection{Postprocessing}
%explain beta decay on data -> create new data
%code snippet from beta-decay
%explain 'extraction' to new data-files and subsequent reduction to plots

\subsection{results}
%description of the different experiments
%explain results/plots for each experiment
%explain how they differ

%add plots of Os-187, Re-187, Os-187/Re-187 for regular MCExperiment
%add plots of Os-187, Re-187, Os-187/Re-187 for MCExperiment w/IMFslope
%add plots of Os-187, Re-187, Os-187/Re-187 for regular MCExperiment wo/decay

