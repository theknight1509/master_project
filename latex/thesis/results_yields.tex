\section{Uncertainty of yields}

\subsection{quick recap of purpose and method}
%vary yields, keep everything else constant
The 'Omega' model with 'Eris' bestfit parameters is used to calculate the amount on \re{187}.
A single function of 'Omega' is overwritten in order to change the stellar yields of \re{187}.
The function multiplies the stellar yield of \re{187} ($Y_\re{187}$) with a constant, the new yield is denoted $\hat{Y}_\re{187}$.

The yield table for binary neutron star mergers is taken from \cite{arnould07} and is the calculated distribution of r-process isotopes in the \sos. The \sos-distribution of isotopes is measured from the solar photosphere and chondrite meteorites in \cite{landolt93}. The s-process distribution can be calculated from nuclear reaction networks, and the r-process/p-process distributions are calculated by subtracting the s-process distribution after fitting to the \sos-abundances.
The basic assumption is that all neutron star mergers eject material with the same isotopic distribution, and this distribution matches the observed solar distribution.
In short, the yield-table used is the \textit{observed} r-process distribution of the \sos, and this distribution is applied to every neutron star mergers.

In \eris-postprocessing\cite{shen15} the yield-table applied to all neutronstar mergers is a different one. The yield table was based on simulations by Rosswog\comment{(Need some reference?)} of neutron star mergers. Those yield-tables however does not contain any output of \re{187}, which is essential for the purpose of this thesis.

\textit{What are the underlying uncertainties of the yield-tables that we so generously apply to our simulations?}
The yield-tables represent the distribution of isotopes and elements ejected from a star during the end of it's lifetime. This information comes from stellar evolution codes like \cite{paxton11} and the nuclear reaction networks applied to them, like \cite{pignatari16}.
Uncertainties in nuclear reaction rates, stellar interior environments and ejecta composition all add together to create a total uncertainty of yield tables which are hard to separate.
In this case, the neutron star mergers are \textit{assumed} to be the dominant contributer to r-process isotopes in the galaxy, and the distribution is \textit{assumed} to follow observed r-process distribution in our \sos (this is ultimately an assumption of homogeneity).
Following these assumptions, the uncertainties in nucleosynthesis of r-process isotopes will simply be the uncertainty \sos-observations\cite{arnould07}.
Simulations suggest\comment{(some reference to Rosswog)} that the ejecta yield of binary neutron star mergers are somewhat dependant on the electron-fraction of the initial neutron stars, not taking inhomogeneities, rotation, and magnetic field of the initial neutron stars into account. Also given the rarity of kilonovas (compact object mergers) the assumption of homogeneity is questionable.

%write description and code to single-yield modification

\label{sec:mod-omega}

In order to manipulate the program \omegamodel\ without changing the source code, which can be found on \comment{website to omega}, inheritance of python-classes was used.

By creating a new python-class that inherits all methods from \omegamodel\ (which inherits many methods from \chemevol ), and making a new function with the same name as the function that sets the yield tables, the old funciton is overridden by the new.
The old function in \chemevol\ is called \verb|__set_yield_tables|, and the new function, \verb|chem_evol__set_yield_tables|, has all the same content and overrides the old function. By adding a few lines of code to the end of the yield-table function, a 'fudge factor' is multiplied to the yield of a single isotope across all yield-tables used.
The extra lines of code are shown in listings \ref{lst:mod-omega}.

\begin{lstlisting}[style=custompython, caption={\label{lst:mod-omega}Snippet of code added to the existing function \texttt{\_\_set\_yield\_tables} in \chemevol\ in \omegamodel-framework. The code-snippet multiplies the yield of isotope \texttt{self.experiment\_isotope} with a factor \texttt{self.experiment\_factor} for all yield-tables where the isotope can be found.}]
####################################################
### End of function as written in 'chem_evol.py' ###
""" 
Change ytables(multiply yields of 'isotope' with 'factor') with
value of self.experiment_factor to isotope corresponding to self.experiment_isotope
"""
####################################################

#AGB + massive stars, and pop3 stars
#loop over the different objects
for table_object, table_name in zip([self.ytables, self.ytables_pop3], ["agb/massive", "pop3"]):
    #get list of available metalicities
    loa_metallicities = table_object.metallicities
    for Z in loa_metallicities:
        #get list of masses for each metallicity
        loa_masses = table_object.get(Z=Z, quantity="masses")
        for M in loa_masses:
            #get current yield
            try:
                present_yield = table_object.get(M=M, Z=Z, quantity="Yields",specie=self.experiment_isotope)
            except IndexError: #isotope doesn't exist for this table
                continue
            #modify yield by some factor
            new_yield = present_yield*self.experiment_factor
            #"insert" new yield back into table
            table_object.set(M=M, Z=Z, specie=self.experiment_isotope, value=new_yield)

# SN1a, NS-NS merger, BH-NS merger
# get index of isotope
index_iso = self.history.isotopes.index(self.experiment_isotope)
#loop over different objects
for table_object, table_name in zip([self.ytables_1a, self.ytables_nsmerger, self.ytables_bhnsmerger], ["sn1a", "nsm", "bhnsm"]):
    #get list of available metalicities
    loa_metallicities = table_object.metallicities
    #loop over metallicities
    for i_Z, Z in enumerate(loa_metallicities):
        #get current yield
        try:
            present_yield = table_object.yields[i_Z][index_iso]
        except IndexError: #this means that isotope doesn't exist for this table
            continue
        #modify yield by some factor
        new_yield = present_yield*self.experiment_factor
        #"insert" new yield back into table
        table_object.yields[i_Z][index_iso] = new_yield
\end{lstlisting}


The isotopic uncertainties of the observed s-process and r-process in the \sos\ are taken from \cite{arnould07} and \cite{landolt93}
\importantcomment{Add table with uncertainties}

%Make a table with the r-process quantities of the sun from Arnould et. al. and add calculations.
\section{r-process standard deviation from \sos-aundances}
Most elements heavier than iron in the solar system can be attributed to synthesis by the slow neutron capture process or rapid neutron capture process. Since the astrophysical sites of s-process nucleosynthesis is relatively well known the isotopic distribution can be well approximated by simulations and nuclear reaction networks.

By measuring the isotopic content of the solar system (from photosphere measurements and meteorites that are similar to the solar distribution), the total contribution from s-process and r-process is observed. Scaling the well known s-process distribution to s-only isotopes\footnote{Isotopes which are shielded from r-process nucleosynthesis and produced solely through the s-process.}, gives the distribution of r-process isotopes\mycite{arnould07}.

\begin{table}
  \begin{tabular}{lrrrrr}
\hline
 isotope   &   standard &    min &    max &   $\sigma_{lower}$ &   $\sigma_{upper}$ \\
\hline
 Re-187    &     0.0318 & 0.027  & 0.0359 &            -0.1509 &             0.1289 \\
 Re-185    &     0.0151 & 0.011  & 0.0176 &            -0.2715 &             0.1656 \\
 Os-188    &     0.0707 & 0.0633 & 0.0781 &            -0.1047 &             0.1047 \\
 Os-189    &     0.103  & 0.0961 & 0.109  &            -0.067  &             0.0583 \\
 Os-190    &     0.152  & 0.137  & 0.168  &            -0.0987 &             0.1053 \\
 Os-192    &     0.273  & 0.252  & 0.289  &            -0.0769 &             0.0586 \\
 Eu-151    &     0.0452 & 0.0267 & 0.0482 &            -0.4093 &             0.0664 \\
 Eu-153    &     0.0495 & 0.046  & 0.0526 &            -0.0707 &             0.0626 \\
\hline
\end{tabular}
  \caption[Observed r-process abundances in \sos\ (\mycite{arnould07})]{\label{tab:rncp-yield-sigma}
    Table taken from \mycitetwo{arnould07}{table 1}
    $\sigma_{lower}$, $\sigma_{upper}$ are calculated by the relative fraction between standard value and min, max respectively.
    Upper standard deviation is the relative difference between standard value and maximum value, and lower standard deviation is the relative difference between standard value and minimum value.
  }
\end{table}
\FloatBarrier

The issue with table \ref{tab:rncp-yield-sigma} is in the interpretation of the values. If the uncertianty is assumed to be gaussian in nature (which is a difficault assumtion to argue for), the standard value is interpreted as the mean value of the distributions, while the minimum and maximum values are the -1 $\sigma$ and +1 $\sigma$ values of the distributions. From the calculated values in column 4 and 5 in table \ref{tab:rncp-yield-sigma} the sigma values are not equal. There is not always a linear relationship between the minimum, standard, and maximum values.
Which value to use? Below, in figure \ref{fig:arnould-gauss-dist}, is a set of gaussian distributions using both  the lower standard deviation, the upper standard deviation, and the average of the two.
In order to see for which isotope the effect is greatest.

From figure \ref{fig:arnould-gauss-dist} the greatest difference lies in \eu{151} and \re{185}. For both isotopes, the lower limit is the greatest uncertainty/standard deviation. By using the greatest standard deviations, more uncertainty in the observations are covered, but it is uncertain if the uncertainties can be extended in both directions of the standard value.
%all gaussian plots
\newlength{\subfiglength}
\setlength{\subfiglength}{0.24\textwidth}

\begin{figure}
  \begin{subfigure}{\subfiglength}
    \centering
    \includegraphics[width=\textwidth]{other_data/arnould_plots/isotope_gaussian_Re-187}
    \caption{Upper, lower, and mean standard deviation for \re{187}.}
  \end{subfigure}
  \begin{subfigure}{\subfiglength}
    \centering
    \includegraphics[width=\textwidth]{other_data/arnould_plots/isotope_gaussian_Re-185}
    \caption{Upper, lower, and mean standard deviation for \re{185}.}
  \end{subfigure}
  \begin{subfigure}{\subfiglength}
    \centering
    \includegraphics[width=\textwidth]{other_data/arnould_plots/isotope_gaussian_Eu-151}
    \caption{Upper, lower, and mean standard deviation for \eu{151}.}
  \end{subfigure}
  \begin{subfigure}{\subfiglength}
    \centering
    \includegraphics[width=\textwidth]{other_data/arnould_plots/isotope_gaussian_Eu-153}
    \caption{Upper, lower, and mean standard deviation for \eu{153}.}
  \end{subfigure}
  \begin{subfigure}{\subfiglength}
    \centering
    \includegraphics[width=\textwidth]{other_data/arnould_plots/isotope_gaussian_Os-188}
    \caption{Upper, lower, and mean standard deviation for \os{188}.}
  \end{subfigure}
  \begin{subfigure}{\subfiglength}
    \centering
    \includegraphics[width=\textwidth]{other_data/arnould_plots/isotope_gaussian_Os-189}
    \caption{Upper, lower, and mean standard deviation for \os{189}.}
  \end{subfigure}
  %final line of plots
  \begin{subfigure}{\subfiglength}
    \centering
    \includegraphics[width=\textwidth]{other_data/arnould_plots/isotope_gaussian_Os-190}
    \caption{Upper, lower, and mean standard deviation for \os{190}.}
  \end{subfigure}
  \begin{subfigure}{\subfiglength}
    \centering
    \includegraphics[width=\textwidth]{other_data/arnould_plots/isotope_gaussian_Os-192}
    \caption{Upper, lower, and mean standard deviation for \os{192}.}
  \end{subfigure}
  \caption[Difference in upper and lower deviation from \mycite{arnould07}]{\label{fig:arnould-gauss-dist}
    Figures a-h show the two different standard deviations calculated from table \ref{tab:rncp-yield-sigma} and their average, plotted as gaussian distributions.
    1.0 on the x-axis represent the standard value, and all distributions have a maximum of 1.0 on the y-axis.
    The plots are merely to visualize the values of table \ref{tab:rncp-yield-sigma} as gaussian probability distributions, and the axes are therefore intentionally unlabelled.
  }
\end{figure}
\FloatBarrier

\section{s-process standard deviation from \sos-aundances}
``The quality of these data has continually improved and the most recent compilation by
Anders and Grevesse [89A] lists 37 elements, determined in the photosphere of the Sun, with errors below
25\%''\mycitetwo{landolt93}{p.197}

Estimated accuracy of elemental osmium is 5\% from CI chondrites\footnote{A type of carbonaceous meteorites with an near solar composition\url{https://en.wikipedia.org/wiki/CI_chondrite}.}\mycitetwo{landolt93}{table2, p.203}.

The number abundance of elemental osmium ($\log N(N(H)\equiv 10^{12}$) is given in log value. The uncertainty of the meteorite abundances is then $\simeq 1.45\%$ and the uncertainty in solar photosphere composition is $\simeq 6.90\%$. The difference between the two observations are $20\%$ with respect to the meteorite number abundance\mycitetwo{landolt93}{table3, p.205}, for other types of meteorites the difference is greater.

The solar system abundances, the fitted s-process distribution, and the resulting r-processes from \mycitetwo{landolt93}{table3, p.205} are included in figure \ref{fig:landholt-figures}.

\setlength{\subfiglength}{0.3\textwidth}
\begin{figure}
  \begin{subfigure}{\subfiglength}
    \centering
    \includegraphics[width=\linewidth]{other_data/arnould_plots/fig7_p211_landolt93_sos.png}
    \caption{The observed solar system distribution of isotopes, plotted in number abundance (scaled to Si) against mass number. Data taken from meteorites and solar photosphere.
    Figure 7, page 211, in article.}
  \end{subfigure}
  \begin{subfigure}{\subfiglength}
    \centering
    \includegraphics[width=\linewidth]{other_data/arnould_plots/fig4_p220_landholt93_sncp.png}            
    \caption{The calculated, and fitted to \sos-abundances, s-process distribution of isotopes. Plotted as number abundance (scaled to Si) times neutron capture cross section against mass number of isotopes.
    Figure 4, page 220, in article.}
  \end{subfigure}
  \begin{subfigure}{\subfiglength}
    \centering
    \includegraphics[width=\linewidth]{other_data/arnould_plots/fig5_p209_landholt93_rncp.png}
    \caption{The r-process abundances, resulting from subtracting calculated s-process abundance from \sos abundance.
    Figure 5, page 209, in article.}
  \end{subfigure}
  \caption{\label{fig:landholt-figures}
    \sos number abundances and derived isotopic distributions from \mycite{landolt93}.
  }
\end{figure}

\FloatBarrier


%\comment{add single uncertainty of %\re{187}, \re{185}, \eu{x}-isotopes, and \os{x}-isotopes} }
%Plots of Eu-151
\setlength{\subfigwidth}{0.5\textwidth}
\begin{figure}
  \begin{subfigure}{\subfigwidth}
    \centering
    \includegraphics[width=0.9\linewidth]{results/plots_yields/mdot_time_Eu-151.png}
    \caption{Time evolution of \eu{151} ejected from supernovae and neutron star mergers.}
  \end{subfigure}
  \begin{subfigure}{\subfigwidth}
    \centering
    \includegraphics[width=0.9\linewidth]{results/plots_yields/ism_time_Eu-151.png}
    \caption{Time evolution of the total mass of \eu{151} in the galaxy over time.}
  \end{subfigure}
  \begin{subfigure}{\subfigwidth}
    \centering
    \includegraphics[width=0.9\linewidth]{results/plots_yields/mdot_hist_Eu-151.png}
    \caption{Sum total of mass of \eu{151} ejected from supernovae and neutron star mergers.}
  \end{subfigure}
  \begin{subfigure}{\subfigwidth}
    \centering
    \includegraphics[width=0.9\linewidth]{results/plots_yields/ism_hist_Eu-151.png}
    \caption{Sum total of mass of \eu{151} in the interstellar medium of the galaxy}
  \end{subfigure}
  \caption{Resulting evolution of \eu{151}-mass in a chemical evolution galaxy model.}
\end{figure}
%table of statistics input uncertainty vs. output uncertainty
\begin{table}
  \input{results/data_yields/table_Eu-151.dat}
\end{table}

%Plots of Eu-153
\begin{figure}
  \begin{subfigure}{\subfigwidth}
    \centering
    \includegraphics[width=0.9\linewidth]{results/plots_yields/mdot_time_Eu-153.png}
    \caption{Time evolution of \eu{153} ejected from supernovae and neutron star mergers.}
  \end{subfigure}
  \begin{subfigure}{\subfigwidth}
    \centering
    \includegraphics[width=0.9\linewidth]{results/plots_yields/ism_time_Eu-153.png}
    \caption{Time evolution of the total mass of \eu{153} in the galaxy over time.}
  \end{subfigure}
  \begin{subfigure}{\subfigwidth}
    \centering
    \includegraphics[width=0.9\linewidth]{results/plots_yields/mdot_hist_Eu-153.png}
    \caption{Sum total of mass of \eu{153} ejected from supernovae and neutron star mergers.}
  \end{subfigure}
  \begin{subfigure}{\subfigwidth}
    \centering
    \includegraphics[width=0.9\linewidth]{results/plots_yields/ism_hist_Eu-153.png}
    \caption{Sum total of mass of \eu{153} in the interstellar medium of the galaxy}
  \end{subfigure}
  \caption{Resulting evolution of \eu{153}-mass in a chemical evolution galaxy model.}
\end{figure}
%table of statistics input uncertainty vs. output uncertainty
\begin{table}
\input{results/data_yields/table_Eu-153.dat}
\end{table}

%Plots of Os-189
\begin{figure}
  \begin{subfigure}{\subfigwidth}
    \centering
    \includegraphics[width=0.9\linewidth]{results/plots_yields/mdot_time_Os-189.png}
    \caption{Time evolution of \os{189} ejected from supernovae and neutron star mergers.}
  \end{subfigure}
  \begin{subfigure}{\subfigwidth}
    \centering
    \includegraphics[width=0.9\linewidth]{results/plots_yields/ism_time_Os-189.png}
    \caption{Time evolution of the total mass of \os{189} in the galaxy over time.}
  \end{subfigure}
  \begin{subfigure}{\subfigwidth}
    \centering
    \includegraphics[width=0.9\linewidth]{results/plots_yields/mdot_hist_Os-189.png}
    \caption{Sum total of mass of \os{189} ejected from supernovae and neutron star mergers.}
  \end{subfigure}
  \begin{subfigure}{\subfigwidth}
    \centering
    \includegraphics[width=0.9\linewidth]{results/plots_yields/ism_hist_Os-189.png}
    \caption{Sum total of mass of \os{189} in the interstellar medium of the galaxy}
  \end{subfigure}
  \caption{Resulting evolution of \os{189}-mass in a chemical evolution galaxy model.}
\end{figure}
%table of statistics input uncertainty vs. output uncertainty
\begin{table}
  \input{results/data_yields/table_Os-189.dat}
\end{table}

%Plots of Os-192
\begin{figure}
  \begin{subfigure}{\subfigwidth}
    \centering
    \includegraphics[width=0.9\linewidth]{results/plots_yields/mdot_time_Os-192.png}
    \caption{Time evolution of \os{192} ejected from supernovae and neutron star mergers.}
  \end{subfigure}
  \begin{subfigure}{\subfigwidth}
    \centering
    \includegraphics[width=0.9\linewidth]{results/plots_yields/ism_time_Os-192.png}
    \caption{Time evolution of the total mass of \os{192} in the galaxy over time.}
  \end{subfigure}
  \begin{subfigure}{\subfigwidth}
    \centering
    \includegraphics[width=0.9\linewidth]{results/plots_yields/mdot_hist_Os-192.png}
    \caption{Sum total of mass of \os{192} ejected from supernovae and neutron star mergers.}
  \end{subfigure}
  \begin{subfigure}{\subfigwidth}
    \centering
    \includegraphics[width=0.9\linewidth]{results/plots_yields/ism_hist_Os-192.png}
    \caption{Sum total of mass of \os{192} in the interstellar medium of the galaxy}
  \end{subfigure}
  \caption{Resulting evolution of \os{192}-mass in a chemical evolution galaxy model.}
\end{figure}
%table of statistics input uncertainty vs. output uncertainty
\begin{table}
  \input{results/data_yields/table_Os-192.dat}
\end{table}

%Plots of Os-190
\begin{figure}
  \begin{subfigure}{\subfigwidth}
    \centering
    \includegraphics[width=0.9\linewidth]{results/plots_yields/mdot_time_Os-190.png}
    \caption{Time evolution of \os{190} ejected from supernovae and neutron star mergers.}
  \end{subfigure}
  \begin{subfigure}{\subfigwidth}
    \centering
    \includegraphics[width=0.9\linewidth]{results/plots_yields/ism_time_Os-190.png}
    \caption{Time evolution of the total mass of \os{190} in the galaxy over time.}
  \end{subfigure}
  \begin{subfigure}{\subfigwidth}
    \centering
    \includegraphics[width=0.9\linewidth]{results/plots_yields/mdot_hist_Os-190.png}
    \caption{Sum total of mass of \os{190} ejected from supernovae and neutron star mergers.}
  \end{subfigure}
  \begin{subfigure}{\subfigwidth}
    \centering
    \includegraphics[width=0.9\linewidth]{results/plots_yields/ism_hist_Os-190.png}
    \caption{Sum total of mass of \os{190} in the interstellar medium of the galaxy}
  \end{subfigure}
  \caption{Resulting evolution of \os{190}-mass in a chemical evolution galaxy model.}
\end{figure}
%table of statistics input uncertainty vs. output uncertainty
\begin{table}
  \input{results/data_yields/table_Os-190.dat}
\end{table}

%Plots of Os-188
\begin{figure}
  \begin{subfigure}{\subfigwidth}
    \centering
    \includegraphics[width=0.9\linewidth]{results/plots_yields/mdot_time_Os-188.png}
    \caption{Time evolution of \os{188} ejected from supernovae and neutron star mergers.}
  \end{subfigure}
  \begin{subfigure}{\subfigwidth}
    \centering
    \includegraphics[width=0.9\linewidth]{results/plots_yields/ism_time_Os-188.png}
    \caption{Time evolution of the total mass of \os{188} in the galaxy over time.}
  \end{subfigure}
  \begin{subfigure}{\subfigwidth}
    \centering
    \includegraphics[width=0.9\linewidth]{results/plots_yields/mdot_hist_Os-188.png}
    \caption{Sum total of mass of \os{188} ejected from supernovae and neutron star mergers.}
  \end{subfigure}
  \begin{subfigure}{\subfigwidth}
    \centering
    \includegraphics[width=0.9\linewidth]{results/plots_yields/ism_hist_Os-188.png}
    \caption{Sum total of mass of \os{188} in the interstellar medium of the galaxy}
  \end{subfigure}
  \caption{Resulting evolution of \os{188}-mass in a chemical evolution galaxy model.}
\end{figure}
%table of statistics input uncertainty vs. output uncertainty
\begin{table}
  \input{results/data_yields/table_Os-188.dat}
\end{table}

%Plots of Re-185
\begin{figure}
  \begin{subfigure}{\subfigwidth}
    \centering
    \includegraphics[width=0.9\linewidth]{results/plots_yields/mdot_time_Re-185.png}
    \caption{Time evolution of \re{185} ejected from supernovae and neutron star mergers.}
  \end{subfigure}
  \begin{subfigure}{\subfigwidth}
    \centering
    \includegraphics[width=0.9\linewidth]{results/plots_yields/ism_time_Re-185.png}
    \caption{Time evolution of the total mass of \re{185} in the galaxy over time.}
  \end{subfigure}
  \begin{subfigure}{\subfigwidth}
    \centering
    \includegraphics[width=0.9\linewidth]{results/plots_yields/mdot_hist_Re-185.png}
    \caption{Sum total of mass of \re{185} ejected from supernovae and neutron star mergers.}
  \end{subfigure}
  \begin{subfigure}{\subfigwidth}
    \centering
    \includegraphics[width=0.9\linewidth]{results/plots_yields/ism_hist_Re-185.png}
    \caption{Sum total of mass of \re{185} in the interstellar medium of the galaxy}
  \end{subfigure}
  \caption{Resulting evolution of \re{185}-mass in a chemical evolution galaxy model.}
\end{figure}
%table of statistics input uncertainty vs. output uncertainty
\begin{table}
  \input{results/data_yields/table_Re-185.dat}
\end{table}

%Plots of Re-187
\begin{figure}
  \begin{subfigure}{\subfigwidth}
    \centering
    \includegraphics[width=0.9\linewidth]{results/plots_yields/mdot_time_Re-187.png}
    \caption{Time evolution of \re{187} ejected from supernovae and neutron star mergers.}
  \end{subfigure}
  \begin{subfigure}{\subfigwidth}
    \centering
    \includegraphics[width=0.9\linewidth]{results/plots_yields/ism_time_Re-187.png}
    \caption{Time evolution of the total mass of \re{187} in the galaxy over time.}
  \end{subfigure}
  \begin{subfigure}{\subfigwidth}
    \centering
    \includegraphics[width=0.9\linewidth]{results/plots_yields/mdot_hist_Re-187.png}
    \caption{Sum total of mass of \re{187} ejected from supernovae and neutron star mergers.}
  \end{subfigure}
  \begin{subfigure}{\subfigwidth}
    \centering
    \includegraphics[width=0.9\linewidth]{results/plots_yields/ism_hist_Re-187.png}
    \caption{Sum total of mass of \re{187} in the interstellar medium of the galaxy}
  \end{subfigure}
  \caption{Resulting evolution of \re{187}-mass in a chemical evolution galaxy model.}
\end{figure}
%table of statistics input uncertainty vs. output uncertainty
\begin{table}
  \centering
  \caption*{title}
  \input{results/data_yields/table_Re-187.dat}
  \caption{caption}
\end{table}

%conclusion - linear relationship between input/output variation
% - most uncertainties come from stellar evolution simulations
\FloatBarrier
