\section{\eris\ simulation}

%% Introduction (ref Guedes11)
\eris\mycite{guedes11e} is a N-body/smooth particle hydrodynamics simulation of a galaxy forming in $\Lambda$CDM cosmology (see section \ref{sec:cosmology}. The simulation is performed with \gasoline\mycite{wadsley03}.

%use \figwidth for width of figure

  \centering
  \includegraphics[width=\figwidth]{img/eris_map_guedes11_fig2.png}
  \caption[\mycitetwo{guedes11e}{fig.2}]{
    \label{img:eris-map}
    \textit{Left panel}:
    Radiative properties of \eris\ from the stellar composition (assuming a Kroupa initial mass function.
    \textit{Right panel}:
    Projected density maps of neutral gas in \eris.

    Both panels show the galaxy face-on and edge-on.
    Plots and figures are taken from \mycitetwo{guedes11e}{fig.2 and references therein}.
  }


%% SPH
\subsection{Smoothed particle hydrodynamics}
%% - particle-sim
Particle physics can be simulated by calculating the forces acting on each particle, then using newtons second law to get the acceleration of each particle.
By integrating in time, with a large variety of numerical methods, the acceleration will give correction to the velocity in the next timestep.
In a similar fashion the velocity of each particle will give the position in the next timestep.
Examples of such numerical methods are Euler's method, Euler-Cromer's method, Velocity Verlet, Runge-Kutta 4.
Numerical particle physics approximates all matter to discrete points in time and space. Due to limited numerical capabilities, and the sheer scale of astronomical dimensions it will not be able to resolve an entire galaxy.
%% - meshgrid-sim
Alternatively to representing a galaxy by many discrete points, is representing a galaxy as a meshgrid of average physical properties.
This meshgrid holds physical properties like density, pressure, velocity and internal energy. Flow of mass and energy between the gridspaces is done according to hydrodynamics.
%% - combination (sphere of gas/kernel, hydrodynamical effects)
Smoothed particle hydrodynamics \mycite{monaghan92} represent an effort to combine the two different approaches. Point particles representing a sphere of gas/stars/dark matter with a kernel function to extend the sphere around the point particle.
The kernel function is historically represented by a gaussian, but for numerical ease modern methods use a spline-function that is zero outside a given range.
Hydrodynamical effects on a particle from another nearby particle is given by the overlapping kernel functions, which represent a higher density/pressure/internal energy region. The hydrodynamical equations then govern change in velocity and temperature.

\noindent
\fbox{
\begin{minipage}{0.5\linewidth}
  Gravity and acceleration is calculated in \gasoline\ by solving the Poisson's equation (eq.\ref{eq:poisson}) and Newton's laws (eq.\ref{eq:newton}) for each body.
  $\phi$ is the gravitational potential, $G$ is the gravitational constant, $\rho$ is the mass density, $a_i$ is the acceleration on body i, $M_j$ is the mass of body j, $r_i - r_j$ is the distance between body j and i.
\end{minipage}
\begin{minipage}{0.5\linewidth}
  \begin{equation}
    \nabla^2\phi = 4\pi G \rho
    \label{eq:poisson}
  \end{equation}
  \begin{equation}
    a_i = \nabla\phi = \sum_{i\neq j} \frac{GM_j}{(r_i-r_j)^2}
    \label{eq:newton}
  \end{equation}
\end{minipage}
}

\noindent
\fbox{
  \begin{minipage}{0.5\linewidth}
  The extension of a single particle is represented by a symmetric kernel function that extends to some smoothing length $h$, with an associated density of (eq.\ref{eq:density}).
  $m_j$ is the mass of particle j, with $W_{ij}$ being the kernel function between particles i and j.
  \end{minipage}
  \begin{minipage}{0.5\linewidth}
    \begin{equation}
      \label{eq:density}
      \rho_i = \sum_{j=1}^{n} m_jW_{ij}
    \end{equation}
  \end{minipage}
}

\noindent
\fbox{
  \begin{minipage}{0.5\linewidth}
  The acceleration from gas pressure between two nearby particles is given by the momentum equation (eq. \ref{eq:momentum}).
  $\vec{v}$ is the velocity of particle i, $P_i$ is the pressure associated with particle i, and $\Pi_{ij}$ is the artificial viscosity term between the two particles.
  \end{minipage}
  \begin{minipage}{0.5\linewidth}
    \begin{equation}
      \label{eq:momentum}
      \frac{\textrm{d}\vec{v}_i}{\textrm{dt}} = - \sum_{j=1}^{n} m_j \left( \frac{P_i}{\rho_i^2} + \frac{P_j}{\rho_j^2} + \Pi_{ij} \right)\nabla_i W_{ij}
    \end{equation}
  \end{minipage}
}

\noindent
\fbox{
  \begin{minipage}{0.5\linewidth}
  The change in temperature/internal energy for the gas in a particle is given by (eq.\ref{eq:internal-energy}).
  $u_i$ is the internal energy of body i and $\vec{v}_{ij}$ is the relative velocity between the two bodies, i and j.  
  \end{minipage}
  \begin{minipage}{0.5\linewidth}
    \begin{equation}
      \label{eq:internal-energy}
      \frac{\textrm{d}u_i}{\textrm{dt}} = \frac{P_i}{\rho_i^2} \sum_{j=1}^{n} m_j \vec{v}_{ij} \dot \nabla_i W_{ij}
    \end{equation}
  \end{minipage}
}

%% - tree structure (map distances, mass-center for far-away distances)
If a group of bodies is far away from an observer, the sum force of each body is similar to force to the mass-center with a mass equal to the sum of indivdual masses.
This is exploited numerically by splitting the point particles into ``trees'' where the force on one body from a group of smaller bodies far away is given by the distance to the mass-center of said group and the sum of masses.
Numerically this is done by calculating the mass-center of all bodies, then splitting the simulation in two sections with equal number of particles and calculate the mass-center of these two sections. This process is repeated iteratively until a single body/point particle is left. In other words; the ``tree'' stretches into many ``branches'' with a single particle at the end of each branch. Nearby particles are also grouped together into ``leaves''.
When calculating the force acting on a single particle one calculates the force from each neighbour, but as the distance increases the mass-center for a large ``branch'' is used instead of all individual ``leaves''.

%% - gasoline (general, cooling, dynamical timestepping, drop layers-paragraph)
\gasoline\ is a parallelized tree-smoothed-particle-hydrodynamics code that tracks particles with a selfgravitating sphere of density, represented by a kernel function around the particle, extending a certain distance from the particle. The particles are ``falling in their own gravity'', while ``flowing their own medium''.
The code builds a tree to connect all the particles, in order to calculate gravitational forces more efficiently. The tree is also reproduced to calculate periodic boundary conditions more efficiently.

Particle hydrodynamics is only relevant when two smoothing kernels (representing the extension of gas) overlap. When the edges of two kernels start overlapping, the density and pressure the kernel represent increase causing a repulsive force from the hydrodynamical equations of motion and continuity.

Energy transport in \gasoline\ can follow a wide range of procedures, like adiabatic and isothermal cooling processes, hydrogen/helium cooling processes with ionization fractions in addition to ultraviolet feedback from star formation and cosmic background.

%% ``Angular momentum problem'' description + solution + sources
\subsection{Angular momentum problem}
In galactic simulations there is an ``angular momentum problem''. This refers to baryonic components having much less rotaional spin in simulations than real observations. This failure was believed to arise from friction moving angular momentum from sub-structures to outer halo when these sub-structures merge causing the cold clumps of gas to fall to the center\mycite{navarro91}.
Meaning that cosmological simulation with gas become clumped in the center of halos, instead of forming disks like observed galaxies. The ``problem'' being not enough angular momentum in the gas.
In newer times this problem have been attempted solved with energy injected from supernovae, meaning evolving stars from the gas content to decrease the effect of cooling and removing angular moment from the center of the galaxy. The stars heat the surrounding gas, and the supernovae add kinetic energy to the gas so it is not allowed to cluster too much in the center.
In the \eris-simulation however, an rotational disk is allowed to form. This is because the high resolution (many particles) combined with the high threshold for star formation centeres the star forming regions in the high density regions. Such compact star formation drives cold gas outflow from the center of the galaxy, removing low angular momentum gas into ``higher orbits''.

%% Fig galaxy-sheet
\setlength{\figwidth}{\linewidth}

\begin{figure}
  \centering
  \includegraphics[width=\figwidth]{img/HubbleTuningFork.jpg}
  \caption[Hubble morphological classification of galaxies]{\label{img:HubbleTuningFork}
    The Hubble morphological classification of galaxies from their observed shape. Described either as elliptical, spherical or spherical with a bar in the center.
    Source: ESA
  }
\end{figure}


%% Details + properties eris
\subsection{Properties of the \eris-simulation}
The simulation consist of dark matter particles and baryonic gas
particles. Star particles are created when the number density
passes 5 atoms cm$^{-3}$\footnote{Assuming three quarter neutral hydrogen and one quarter neutral helium, by mass, this threshold density equals $0.043\msol pc^{-3}$. A gas particle is initially represented by a total mass of $2\times10^{4}\msol$ and a smoothing length/radius of 120pc.}. Feedback from an active galactic nucleus
is neglected, but supernova-feeback is considered along with
cosmic UV background and radiative cooling.

Some properties of the simulated galaxy:
\begin{itemize}
\item{rotaional disk with scale length $R_d=2.5kpc$}
\item{``gentle'' rotation curve with circular velocities up to 2.2
  scale lengths}
\item{i-band (infrared wavelength 806 nm, bandwidth 149 nm)
  bulge-to-disk-ratio of $B/D=0.35$}
\item{baryonic mass fraction inside halo is 30\% lower then
  cosmic average}
\item{thin disk with typical HI-stellar mass-ratio}
\item{disk is forming stars in $\Sigma_{sfr}-\Sigma_{HI}$ plane}
\item{disk falls on photometric Tully-Fisher relation and
  stellar mass - halo virial mass relation}
\item{structural properties, mass budget, and scaling relations between mass and luminosity matches several observational constraints}
\end{itemize}

%use \figwidth for width of figure

\begin{figure}
  \centering
  \includegraphics[width=\figwidth]{img/rotation_curve_guedes_fig1a.png}
  \caption[\mycitetwo{guedes11e}{fig.1}]{
    \label{img:rotation-curve}
    Contribution to rotation curve in \eris\ at redshift zero.
    The circular velocity is calculated as a function of radius, $v_c = \sqrt{GM(<r)/r}$, for the different types of particles.
    \textit{Red dash-dotted line}; gas within virial radius,
    \textit{blue short dashed line}; stars, 
    \textit{green long dashed line}; dark matter, 
    \textit{black solid line}; total contribution from all mass in \eris,
    \textit{data points}; \textit{Sloan Digital Sky Survey}.

    Plots and figures are taken from \mycitetwo{guedes11e}{fig.1 and references therein}.
  }
\end{figure}



%% What eris present (figs?)
\mycite{guedes11e} presents a realistic simulation of a Milky Way type galaxy using a new smooth particle hydrodynamic cosmological simulation called \eris, based on \gasoline\mycite{wadsley03}. It includes radiative cooling, cosmic UV heating, supernova feedback, and high-density star formation requirement (which is believed to be a key ingredient in producing realistic spirals in galactic simulations).
The high density threshold for star formation (5 atoms cm$^{-3}$) creates sparsely, clustered regions of star formation that remove low-angular momentum cold gas from the center of the halo. 

By using the \eris-simulation, the chemical evolution of the milky way is studied.
\eris\ traces oxygen and iron from supernovae.
\mycite{shen15} postprocesses the simulation data from \eris\ to include neutron star mergers that add europium (an r-process tracer) to the interstellar medium.
The study shows that the heavy products of neutron star mergers can be incorporated into early stars, even if the shortest neutron star mergers is 100 Myr. The conclusion of the study does not vary much with delay-time and merger rate and an argument is made for neutron star mergers being the dominant r-process source in the galaxy.

This is supported by the neutron star merger (GW170817) observed\comment{ cite gw170817}\mycite{gw170817}.
With \textit{estimated} dynamical ejecta of mass 0.002\msol and velocity of 0.2c and highspeed wind ejecta of mass 0.015\msol and velocity of 0.08c. Such velocities are high enough to cross a galaxy within $\simeq 1$ Gyr, and if most of the ejecta manifests as r-process material the assumed model of \mycite{shen15} (ejected r-process material of 0.01\msol) is realistic. The single GW170817 observation makes rate of such events hard to estimate, and therefore difficult to constrain the neutron star merger rate in \eris.

%% Move stuff to theory I section
\subsection{move to theory I}
Looking at very metal poor stars in our Galaxy, which have been around for a long time, r-process abundances can be found. Meaning that the source of r-process has been around for a long time, and in a robust manner. However, the large variations show that the process was unhomogenous for early times, while it is more smoothed after many Galactic rotations and repeated events.
The two main sources of producing these heavy r-process elements are believed to be in the merger of two neutron stars (or the merger between a neutron star and a black hole) or in a heavy core collapse supernova. The production yields are much larger for neutron star mergers, but they are also much more rare.
\comment{(Important citations Takashi94 and Woosley94 for SNII; Lattimer 77 and Freiburghaus 99 for NSM)}

r-process nucleosynthesis requires neutron heavy isotopes, and the two leading theories are neutron star mergers and type II supernovae (see references Burbridge 1957, Roberts 2010, Lattimer 1977). Even though the conditions of the neutron star environment are somewhat uncertain, estimates are promising for the neutron star mergers to produce heavy isotopes in r-process distributions.
These two processes, neutron star mergers and type II supernovae are quite different in frequency and yields, meaning that galactic chemical evolution models should be able to predict which of the models are most likely.

\subsection{neutron star merger application to \eris}
Numerically, the neutron star mergers are described by delay-time distribution, merger rate, yield of r-process elements, and the spatial distribution of events.
The delay-time distribution is modelled by a power-law, $P(t) \propto t^{-n}$, from some minimum timescale to the hubble-time(end of simulation).
Each neutron star merger is assumed to synthesize some mass of r-process material, only a fraction of this material will be europium(which is used as the tracer).
The ratio of europium to r-process material is usually assumed to be solar\comment{($M_{Eu}=9.3\times10^{-4}M_{r-process}$ Sneden 2008)}, while the merger rate is calculated from scaling the star formation integral until europium-oxygen ratio equals solar ratio.

From the merger rate, the number of injection events per timestep is calculated and applied to random star particles in the data. The ejection material is then distributed among gas particles in close proximity to the selected star particles.

%% chemical observations
\subsection{Chemical observations of \eris}
At redshift zero the oxygen-iron abundances can be split into two main sources.
One source is type II supernovae, which are more rich in oxygen, leading to higher (supersolar) ratios.
Another source is type Ia supernovae, which are more rich in iron, leading to more iron than oxygen.

There are two main implementations of Eris postprocessing involved, one without any mixing, and another with mixing of metals between gas particles. For both oxygen-iron ratios and europium-iron ratios one sees that mixing gives less variation between ``upper'' and ``lower'' sigma-bands.

The chemical enrichment is closely tied to the star formation and evolution. \eris\ is a good approximation to the Milky Way galaxy do to similar star formation, shape and chemical composition today.
\mycite{shen15} finds that neutron star mergers are capable of enriching the surrounding medium, even with a minimum delay-time of 100 Myr.

The dispersion of [Eu/Fe] is big enough, even at low metallicities, to reproduce the variations of observed europium in old low-mass stars.
The mixing level affects the abundance of europium, but it is hard to compare to observations becasue spectroscopic abundance of many stars are unknown.
The conclusion is that variations of the model parameters do not significantly alter the result.

Galactic chemical evolution models, like \omegamodel, are single points in space with mass resolution and time-integration. These models are simple way of calculating the mean amount of elements in the galaxy based on a star formation history, yield tables and initial composition.
These models do not replicate the inhomogeneities and variations in metal-distributions that are found in N-body simulations.
\mycite{shen15} attempts to reproduce the results with a 1D-model based on the parameters found in \eris;
At late times model agrees well with the average of all of \eris, however it does not agree well with the early results of Eris, nor does it replicate the large variations in spectroscopic abundance during early times.

%use \figwidth for width of figure

\begin{figure}
  \centering
  \includegraphics[width=\figwidth]{img/spectroscopic_eu_shen15_fig4.png}
  \caption[\mycitetwo{shen15}{fig.4}]{
    \label{img:spectroscopic-eu}
    Top two panels show spectroscopic Eu in star particles against the formation time of said star particles. Time is shown in log-scale.
    The blue lines show mean, 68\%, and 95\% intervals of the stars.
    The bottom two panels show spectroscopic Eu in star particles relative to the iron abundance in said star particles.
    Dashed line represent the solar value, while shaded regions are the 68\% and 95\% intervals of the stellar data.
    All data is retrieved from a representative subsample of star particles at the end of the simulation (redshift zero).

    The two left panels show the data from the \eris-simulation when there is no metallicity floor (a metallicity floor means that all particles have a minimum amount of metals at the beginning of the simulation, usually $10^{-4}\times$solar metallicity) and no mixing of metals between gas particles.
    On the two right panels, a metallicity floor and mixing of metals are implemented. This effect appears to lower the spread of metals in star particles at all times, and this result is consistent for iron and oxygen too\mycite{shen15}.

    Plots and figures are taken from \mycitetwo{shen15}{fig.4}
  }
\end{figure}

%use \figwidth for width of figure

  \centering
  \includegraphics[width=\figwidth]{img/spectroscopic_abundance_shen15_fig6.png}
  \caption[\mycitetwo{shen15}{fig.6}]{
    \label{img:spectroscopic-abundance-eris}
    Top panel show [O/H] against log time in \eris. Oxygen is used as a tracer for $\alpha$ elements produced in hydrostatic burning.
    The middle panel show [Fe/H] against log time in \eris. 
    Bottom panel show [Eu/H] against log time. Europium being produced in neutron star mergers randomly distributed among the star particles in \eris.
    The buildup of metals is represented after mixing of metals and a metallicity floor is introduced.

    The black dashed line represent solar ratios, while the green shaded regions are 68\% and 95\% intervals of stars born at that time.
    The blue line shows the average metallicity of cold gas, which is equivalent to a one-zone galactic chemical evolution model.
    It is clear that the global average of cold gas can not reproduce the metallicity of early times.
    
    Plots and figures are taken from \mycitetwo{shen15}{fig.4}.
  }


\iffalse
\importantcomment{This section is outdated as of 10.05.18}

\section{\eris simulation}

'Eris' is a N-body/smooth particle hydrodynamics simulation of a
galaxy forming in $\Lambda$CDM cosmology.
The simulation consist of dark matter particles and baryonic gas
particles. Star particles are created when the number density
passes 5 atoms cm$^-3$. Feedback from an active galactic nucleus
is neglected, but supernova-feeback is considered along with
cosmic UV background and radiative cooling.

Some properties of the simulated galaxy:
\begin{itemize}
\item{rotaional disk with scale length $R_d=2.5kpc$}
\item{``gentle'' rotation curve with circular velocities at 2.2
  scale lengths}
\item{i-band (infrared wavelength 806 nm, bandwidth 149 nm)
  bulge-to-disk-ratio of $B/D=0.35$}
\item{baryonic mass fraction inside halo is 30\% lower then
  cosmic average}
\item{thin disk with typical HI-stellar mass-ratio}
\item{disk is forming stars in $\Sigma_{sfr}-\Sigma_{HI}$ plane}
\item{disk falls on photometric Tully-Fisher relation and
  stellar mass - halo virial mass relation}
\item{structural properties, mass budget, and scaling relations between mass and luminosity matches several observational constraints}
\end{itemize}

In galactic simulations there is an ``angular momentum problem''. This refers to baryonic components having much less rotaional spin in simulations than real observations. This failure was believed to arise from friction moving angular momentum from sub-structures to outer halo when these sub-structures merge causing the cold clumps of gas to fall to the center.
In newer times this problem have been attempted solved with energy injected from supernovae, meaning evolving stars from the gas content to decrease the effect of cooling and removing angular moment from the center of the galaxy.
Star formation in the disk comes from inflow of cold baryonic gas that was never shock heated to virial temperature.
\comment{INSERT S0...-GALAXY-SHEET.}
%link thesis/img/HubbleTuningFork.jpg

Yet the simulated galaxies have more centered baryon components and reproduce only S0 and Sa type galaxies. With two major exceptions there are no simulations of type Sb and Sc, one exception with low star formation and another with low mass.

This paper presents a realistic simulation of a Milky Way type galaxy using a new smotth particle hydrodynamic cosmological simulation. It includes radiative cooling, cosmic UV heating, supernova feedback, and high-density star formation requirement(which is believed to be a key ingridient.

The high threshold for star formation is important to create non-centered galaxies

%r-process post-production with Shen 2015
In Shen 15 \comment{(insert proper citation)} the simulation data from 'Eris' is post-processed to include, not only oxygen adn iron, but also europium from neutron star mergers.

By using the 'Eris' simulation\cite{guedes11e}, the chemical evolution of the milky way is studied.
'Eris' traces oxygen and iron from supernovae and in this work, postproduction traces neutron star mergers and the europium ejected from them post-merger.
r-process abundance is traced in the milky way proxy by the [Eu/Fe]-ratio.
The study shows that the heavy products of neutron star mergers can be incorporated into early stars, even if the shortest neutron star mergers is 100 Myr.

The conclusion of the study does not vary much with delay-time and merger rate and an argument is made for neutron star mergers being the dominant r-process source in the galaxy.

%\section{introduction}
Looking at very metal poor stars in our Galaxy, which have been around for a long time. r-process abundances can be found. Meaning that the source of r-process has been around for a long time, and in a robust manner. However, the large variations show that the process was unhomogenous for early times, while it is moore smoothed after many Galactic rotations and repeated events.
The two main regions of producing these heavy r-process elements are in the merger of two neutron stars (or the merger between a neutron star and a black hole) or in a heavy core collapse supernova. The production yields are much larger for neutron star mergers, but they are also much more rare.
(Important citations Takashi94 and Woosley94 for SNII; Lattimer 77 and Freiburghaus 99 for NSM)

%\section{methods}
%\subsection{the Eris simulation}
%Hæ?!
%\subsection{R-process production sites and injection history}
The neutron star mergers are described by delay-time distribution, merger rate, yield of r-process elements, the spatial distribution of events.
The delay-time distribution is modelled by a power-law, $P(t) \propto t^{-n}$, from some minimum timescale to the hubble-time(end of simulation).
Each neutron star merger is assumed to created some mass of r-process material, only a fraction of this material will be europium(which is used as the tracer).
The ratio of europium to r-process material is assumed to be solar(Sneden 2008), while the merger rate is calculated from scaling the star formation integral until europium-oxygen ratio equals solar ratio.

The neutron star merger events are set to occur near the stellar distribution, and since the kinetic energy outout is not large compared to supernovae the gas dynamics is unaffected.
In simple terms, the neutron star mergers are injected in stellar regions and therefor drown in the bright, explosive environment of larger supernovae.

Using the time evolution of the star formation rate, the neutron star merger events are injected at random star-particles (simple stellar populations).

%\section{r-process enrichment in the milky way}
At redshift zero the oxygen-iron abundances can be split into two main regions. One primarily enriched by type II supernovae, which are more rich in oxygen, leading to higher (supersolar) ratios. Another which are primarily enriched by type Ia supernovae, leading to more iron than oxygen.

There are two main implementations involved, one without any mixing, and another with mixing of metals between gas particles. For both oxygen-iron ratios and europium-iron ratios one sees that mixing gives less variation between ``upper'' and ``lower'' sigma-bands.

Populating some star particles with neutron star mergers and have them enrich the nearby gas particles, and subsequentually the new star particles, gives a more complete abundance-pattern to trace.
The abundances traced are hydrogen, oxygen (which primarily follows type II supernovae), iron (produced more abundently in type Ia supernovae) and europium (produced in neutron star mergers only). The europium-iron ratio varies widely, even for early times.

%\section{discussion}
r-process nucleosynthesis requires neutron heavy isotopes, and the two leading theories are neutron star mergers and type II supernovae (see references Burbridge 1957, Roberts 2010, Lattimer 1977). Even though the conditions of the neutron star environment are somewhat uncertain, estimates are promising for the neutron star mergers to produce heavy isotopes in r-process distributions.
These two processes, neutron star mergers and type II supernovae are quite different in frequency and yields, meaning that galactic chemical evolution models should be able to predict which of the models are most likely.

The chemical enrichment is closely tied to the star formation rate/history/birth/death, and thereby makes the 'Eris' simulation a good approximation for the Milky Way Galaxy.
This study (shen15 'Eris' rncp post-production) finds that neutron star mergers are capable of enriching the surrounding medium, even with a minimum delay-time of 100 Myr.

%\subsection{dependence on model parameters}
The dispersion/variantion of [Eu/Fe] is great enough, even at low metallicities.
The results changing the parameters in the fiducial model is obvious, and I've elaborated on this before.
The conclusion is that variations of the model parameters do not significantly alter the result.
The mixing level affects the abundance of europium, but it is hard to compare to observations becasue spectroscopic abundance of many stars are unknown.

%\subsection{comparison with 1d models}
Galactic chemical evolution models are single points in space with mass resolution and time-integration. These models are simple way of calculating the mean amount of elements in the galaxy based on a star formation history, yield tables and initial composition.
These models do not replicate the inhomogeneities and variations in metal-distributions.
An attempt is made in this study to reproduce the results with a 1D-model based on the parameters used in Eris.
At late times model agrees well with the average of all of Eris, however it does not agree well with the early results of Eris, nor does it replicate the large variations in spectroscopic abundance during early times.

\fi
