\section{Stellar evolution \comment{all of this is taken from Caroll\&Ostlie chapters 12, 13, and 15}

Regions of space with higher density then their surroundings are called giant molecular clouds and can extend as wide as \comment{insert reference} and are the birth place of stars.

Some regions of these giant molecular clouds will have even larger overdensities and gravity dictates that these overdense regions will eventually fall in on themselves.

A star is a sphere of gas with high enough density, and subsequentally high enough temperature, to maintain stable fusion processes in the core.
If a subregion of a giant molecular cloud is overdense and will eventually collapse to a star, the subregion is known as a protostar.

How large such a subregion must be to collapse is given by the Jeans criterion. The virial theorem states that for a gas in equilibrium the relation between kinetic energy from thermal motion, $E_k$, and potential energy from gravitational collapse, $E_p$, is given by: $2E_k + E_p = 0$.
This means that when a cloud of gas collapses, this equilibrium no longer holds and the gravitational potential energy is greater then the thermodynamical kinetic energy. This unequilibrium is called the Jeans criterion.
For a spherically symmetric gas, with no rotation, magnetic fields, turbulence or pressure from outside forces the mass of the subregion must exceed the Jeans mass:
$M_J = \left(\frac{5kT}{G\mu m_H}\right)^{3/2}\left(\frac{3}{4\pi \rho_0}\right)^{1/2}$
or the region must cover a smaller volume then covered by the Jeans radius:
$R_J = \left(\frac{15kT}{4\pi G\mu m_H\rho_0}\right)^{1/2}$
where k and T is the thermal energy, G is the newtonian gravitational constant, $\mu$ and $m_H$ is the average moleular weight, and $\rho_0$ is the initial density of the subregion.

When temperatures increase, some of the heavier elements will ionize and the free electrons bonds with hydrogen. The $H^-$ ions increase the opacity drastically trapping the heat from gravitational potential energy more efficiently.

Including an external gas pressure, $P_0$, gives the Bonnor-Ebert mass criterion:
$M_{BE} = \frac{c_{BE}\frac{kT}{\mu m_H}}{P_0^{1/2}G^{3/2}}$
where $c_{BE}=1.18$ is a dimensionless constant, and all other variables are given in the text.

Assuming that any pressure-gradient inside the gas is too small to affect the dynamics and that all the gravitational potential energy released is effectively radiated away, making the gas isothermal, all parts of the gas will collapse to a single point at the same time\footnote{Naturally the gas can't collapse to a singularity, but it will collapse to a radius very small compared to the original radius.}. This kind of collapse is called homologous collapse and the free-fall time when all gas reaches the ``singular point'' is given by:
$t_{ff} = \left(\frac{3\pi}{32G\rho_0}\right)^{1/2}$ where all variables are given previously in this section.

Any inhomogeneities in density, and pressure gradient can cause fracturing of the collapsing gas, as can the presence rotation, turbulence, and magnetic fields. Meaning the subregion is divided into smaller regions where the density might not be big enough or several protostars can be created.
This might also lead to binary systems or systems with several stars.
The entire giant molecular cloud can also be considered a collapsing gas, but fracturing causes several stars to be born as separate entities inside.
