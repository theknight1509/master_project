\begin{frame}
\frametitle{Methods}
\begin{itemize}
\item Fitting \omegamodel\ to data from \eris
\item Manipulate yields in \omegamodel
\item Main experiments \todo[rewrite this]
\item Postprocessing
\end{itemize}
\end{frame}

\begin{frame}
\frametitle{Fitting \omegamodel\ to data from \eris}
\begin{itemize}
\item \todo[rough model]
\item \todo[chi2-by-eye]
\item \todo[data available]
\item Steps
\end{itemize}
\end{frame}

\frame{\comment{Direct Insertion}}
\frame{\comment{Mass}}
\frame{\comment{Stellar parameters}}
\frame{\comment{Neutron star mergers}}
\frame{\comment{Time steps}}
\frame{\comment{Final model}}

\begin{frame}
\frametitle{Manipulate yields in \omegamodel}
\begin{itemize}
\item Yields from arnould and other \todo
\item Fudge-factors \todo
\item Linear relationship
\end{itemize}
\end{frame}

\begin{frame}
  \frametitle{Table of observed abundances}
  \begin{table}[h]
    \centering
    \begin{tabular}{lrrrrr}
\hline
 isotope   &   standard &    min &    max &   $\sigma_{lower}$ &   $\sigma_{upper}$ \\
\hline
 Re-187    &     0.0318 & 0.027  & 0.0359 &            -0.1509 &             0.1289 \\
 Re-185    &     0.0151 & 0.011  & 0.0176 &            -0.2715 &             0.1656 \\
 Os-188    &     0.0707 & 0.0633 & 0.0781 &            -0.1047 &             0.1047 \\
 Os-189    &     0.103  & 0.0961 & 0.109  &            -0.067  &             0.0583 \\
 Os-190    &     0.152  & 0.137  & 0.168  &            -0.0987 &             0.1053 \\
 Os-192    &     0.273  & 0.252  & 0.289  &            -0.0769 &             0.0586 \\
 Eu-151    &     0.0452 & 0.0267 & 0.0482 &            -0.4093 &             0.0664 \\
 Eu-153    &     0.0495 & 0.046  & 0.0526 &            -0.0707 &             0.0626 \\
\hline
\end{tabular}}
    \caption[Values and uncertainties of observed r-process abundances in the \sos]{\label{tab:arnould-rncp-uncertainty}
      Values and uncertainties of r-process nuclei near \re{187} from \mycite{arnould07}.
      The relative uncertainty, $\sigma$-values, are calculated on the assumption that min/max are the one-sigma standard deviations in either direction.
    }
  \end{table}
\end{frame}

\begin{frame}
  \frametitle{Chemical evolution of \re{187}}
  \setlength{\subfigwidth}{0.45\textwidth}
  \begin{figure}[h]
    \begin{subfigure}{\subfigwidth}
      \centering
      \includegraphics[width=0.9\linewidth]{\thesisdir{results/plots_yields/mdot_time_Re-187.png}}
      %\caption{Time evolution of \re{187} ejected from supernovae and neutron star mergers.}
    \end{subfigure}
    \begin{subfigure}{\subfigwidth}
      \centering
      \includegraphics[width=0.9\linewidth]{\thesisdir{results/plots_yields/ism_time_Re-187.png}}
      %\caption{Time evolution of the total mass of \re{187} in the galaxy over time.}
    \end{subfigure}
    \begin{subfigure}{\subfigwidth}
      \centering
      \includegraphics[width=0.9\linewidth]{\thesisdir{results/plots_yields/mdot_hist_Re-187.png}}
      %\caption{Sum total of mass of \re{187} ejected from supernovae and neutron star mergers.}
    \end{subfigure}
    \begin{subfigure}{\subfigwidth}
      \centering
      \includegraphics[width=0.9\linewidth]{\thesisdir{results/plots_yields/ism_hist_Re-187.png}}
      %\caption{Sum total of mass of \re{187} in the interstellar medium of the galaxy}
    \end{subfigure}
    %\caption[effect of nucelar uncertainties]{Resulting evolution of \re{187}-mass in a chemical evolution galaxy model.}
  \end{figure}
\end{frame}

\begin{frame}
  \frametitle{Statistical deviation of \re{187}}
  \begin{table}
    \centering
    \input{\thesisdir{results/data_yields/table_Re-187.dat}}
    %% \caption[effect of nuclear uncertainties]{\label{tab:re187-yield-uncertainties}
    %%   Comparison of output mass-uncertainty with input yield-uncertainty.
    %%   $\sigma_{init}$ is the relative uncertainty from table \ref{tab:arnould-rncp-uncertainty} for zero, one, and two sigmas in either direction.
    %%   $\sigma_{ISM}$ is the relative deviation between the resulting mass of the interstellar medium, at current time (Z=0) and integrated for all time ($\Sigma$).
    %%   $\sigma_{\dot{m}}$ is the relative deviation between the ejected mass per timestep into the interstellar medium, at current time (Z=0) and integrated for all time ($\Sigma$).
    %% }
  \end{table}
\end{frame}

\begin{frame}
\frametitle{Main experiments \todo[rewrite this]}
\begin{itemize}
\item Draw random ``fudge-factor'' from gaussian distribution
\item 1500 individual calculations
\item \textbf{Yields}
\item \textbf{Yields+IMFslope}
\item \textbf{Yields+IMFslope+NSM}
\end{itemize}
\end{frame}

\begin{frame}
  \frametitle{Postprocessing}
  \begin{minipage}[t][1cm][t]{0.45\linewidth}
    \betadecay
    \begin{itemize}
      \item $\Delta \mathrm{Re} = -\lambda_{\scriptscriptstyle \mathrm{Re}} \mathrm{Re} \Delta t$
      \item $\Delta \mathrm{Os} = \lambda_{\scriptscriptstyle \mathrm{Re}} \mathrm{Re} \Delta t$
    \end{itemize}
  \end{minipage}
  \hfill
  \begin{minipage}[t][1cm][t]{0.45\linewidth}
    Removing negative negative yields
    \begin{itemize}
      \item $\hat{Y} \leq 0 \rightarrow$ \begin{tabular}{l}Do not \\ consider \\ calculation \end{tabular}
    \end{itemize}
  \end{minipage}
\end{frame}
