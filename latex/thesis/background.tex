%NOTE! requires mhchem-package and self-made comment definitions

%\newcommand{\re187}{\ce{^{187}_{75}Re}}
%\newcommand{\os187}{\ce{^{187}_{76}Os}}

\section{Background}

\subsection{Early universe}
After the big bang, all space and matter existed as a near singular point of near infinite density and temperature (although temperature is hard to define at such scales).
At this point all particles exist in a hot soup of fundamental particles, leptons, quarks, and force-carriers.
\comment{add table 30.2 and reference to Carroll\&Ostlie?\\}
As the universe expands, the temperature and density decreases.
\comment{Inflation\\}
\comment{Formation of baryons/mesons\\}
\comment{Recombination -> neutral universe -> CMB}

\subsubsection{Primordial matter}
\comment{(reference to Caroll and Ostlie?) \\}
\subsubsection{Gravitational collapse}
\subsubsection{The formation of stars and galaxies}
\subsubsection{Galaxies}
\subsubsection{Stars}

\subsection{Nuclear physics of stars}
\subsubsection{The nucleus}
\comment{Write about the modern models for the nucleus}
\subsubsection{Interaction of nuclei}
\comment{nuclear physics, nuclear physics experiments, cross-sections, density and temperature dependence}
\subsubsection{Production of heavy elements}
\comment{Write about the neutron capture processes}
\comment{Make a new section here where I describe the application to stars and chemical production}
\comment{burbridge paper?}

\subsection{Galactic chemical evolution}
\comment{Start by piecing together the tiniest of nuclear physics with the galactic and stellar physics}
\comment{Typical history of chemical enrichment as a galaxy ages}
\comment{Write about models for chemical evolution, both one-zone and others}
\subsubsection{More subsections about GCE \\ Omega \\ Eris}

\subsection{Cosmic clocks}
\importantcomment{I should probably start here!}\\
\re{187} is a long-lived radioactive nucleide with a half-life of $\simeq40$Gyr (\comment{site source for half-life}). This means that, given an cosmic abundance of \re{187}, half of the \re{187}-nuclei will \betadecay to \os{187}. In such a radioactive process, \re{187} is called the parent-nuclide, while \os{187} is called the daughter-nuclide. Since this relation between parent and daughter is exponential in time, the age of the parent can be calculated from the amount of parent-nuclei and daughter-nuclei that decay from parent.
\begin{align*}
  A_x(t) &= A_x(t_0) e^{-\lambda_x t} 
  \qquad \text{ \footnotesize
      \betadecay of radioactive nuclei
  } \\
  \frac{A_x(t)}{A_x(t_0)} &= e^{-\lambda_x t} \\
  \frac{1}{2} &= e^{-\lambda_X \halflife}
  \qquad\qquad \text{\footnotesize 
      insert definition of half-life \halflife} \\
  \lambda_x &= \frac{-\ln\frac{1}{2}}{\halflife}
  = \frac{\ln 2}{\halflife} \\
  A_x(t) &= A_x(t_0) e^{-\frac{\ln 2}{\halflife} t} \\
  \comment{add description of variables}
\end{align*}
\subsubsection{Radioactive isotope dating with C-14 etc.}
A well known example of radioactive dating is the \betadecay from \isotope{C}{6}{14} to \isotope{N}{7}{14}. The \halflife of \isotope{C}{6}{14} is (\comment{insert value and citation here}), and this isotope is created in the atmosphere from cosmic rays (\comment{insert citation here}). In the atmosphere there will always be a fraction of  \isotope{C}{6}{14} proportional to \isotope{N}{7}{14} and cosmic ray irradiation.
For a few multiples of the carbon-14-halflife(\comment{roughly how many years?}), \isotope{N}{7}{14} and cosmic ray irradiation can be assumed to be constant. Since all living animals/plants eventually get their ``daily dose'' of carbon from the air(either by breathing or eating things that breathe), all living animals/plants consist of equal abundance of \isotope{C}{6}{14} relative to the neutral counterpart \isotope{C}{6}{12}. However this exchange of matter stops as the animal/plant in question dies and the metabolism stops, then the \isotope{C}{6}{14} abundance will \betadecay with a halflife of (\halflife=\comment{insert halflife here}). If the skeleton is well kept, the amount of \isotope{C}{6}{14} can be measured and compared to the present day amount. Using the simple exponential equation in (\comment{add reference to equations}) the time of death can be calculated from the halflife and relative amount of radioactive parent:
$$ t_0 = -\frac{\halflife}{\ln 2} \ln \frac{A_\isotope{C}{6}{14}(t_{\text{now}})}{A_\isotope{C}{6}{14}(t_0)}$$
\comment{add description of variables}
\subsubsection{Radioactive isotope dating on r-process elements}
\comment{Generally how to about the whole shit}
%\subsubsection{The \re187-\os187 system}
