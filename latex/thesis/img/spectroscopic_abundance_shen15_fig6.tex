%use \figwidth for width of figure

\begin{figure}
  \centering
  \includegraphics[width=\figwidth]{img/spectroscopic_abundance_shen15_fig6.png}
  \caption[\mycitetwo{shen15}{fig.6}]{
    \label{img:spectroscopic-abundance}
    Top panel show [O/H] against log time in \eris. Oxygen is used as a tracer for $\alpha$ elements produced in hydrostatic burning.
    The middle panel show [Fe/H] against log time in \eris. 
    Bottom panel show [Eu/H] against log time. Europium being produced in neutron star mergers randomly distributed among the star particles in \eris.
    The buildup of metals is represented after mixing of metals and a metallicity floor is introduced.

    The black dashed line represent solar ratios, while the green shaded regions are 68\% and 95\% intervals of stars born at that time.
    The blue line shows the average metallicity of cold gas, which is equivalent to a one-zone galactic chemical evolution model.
    It is clear that the global average of cold gas can not reproduce the metallicity of early times.
    
    Plots and figures are taken from \mycitetwo{shen15}{fig.4}.
  }
\end{figure}
