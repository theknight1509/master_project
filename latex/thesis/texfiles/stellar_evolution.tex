\section{Stellar evolution}
\comment{This section is summarized from \mycitetwo{carroll2007}{ch.12,13,15}}

Regions of space with higher density then their surroundings are called giant molecular clouds and can extend as wide as \comment{insert reference} and are the birth place of stars.

Some regions of these giant molecular clouds will have even larger overdensities and gravity dictates that these overdense regions will eventually fall in on themselves.

A star is a sphere of gas with high enough density, and subsequentally high enough temperature, to maintain stable fusion processes in the core.

How large such a subregion must be to collapse is given by the Jeans criterion. The virial theorem states that for a gas in equilibrium the relation between kinetic energy from thermal motion, $E_k$, and potential energy from gravitational collapse, $E_p$, is given by: $2E_k + E_p = 0$.
This means that when a cloud of gas collapses, this equilibrium no longer holds and the gravitational potential energy is greater then the thermodynamical kinetic energy. This unequilibrium is called the Jeans criterion.
For a spherically symmetric gas, with no rotation, magnetic fields, turbulence or pressure from outside forces the mass of the subregion must exceed the Jeans mass:
$M_J = \left(\frac{5kT}{G\mu m_H}\right)^{3/2}\left(\frac{3}{4\pi \rho_0}\right)^{1/2}$
or the region must cover a smaller volume then covered by the Jeans radius:
$R_J = \left(\frac{15kT}{4\pi G\mu m_H\rho_0}\right)^{1/2}$
where k and T is the thermal energy, G is the newtonian gravitational constant, $\mu$ and $m_H$ is the average moleular weight, and $\rho_0$ is the initial density of the subregion.

Including an external gas pressure, $P_0$, gives the Bonnor-Ebert mass criterion:
$M_{BE} = \frac{c_{BE}\frac{kT}{\mu m_H}}{P_0^{1/2}G^{3/2}}$
where $c_{BE}=1.18$ is a dimensionless constant, and all other variables are given in the text.

Assuming that any pressure-gradient inside the gas is too small to affect the dynamics and that all the gravitational potential energy released is effectively radiated away, making the gas isothermal, all parts of the gas will collapse to a single point at the same time\footnote{Naturally the gas can't collapse to a singularity, but it will collapse to a radius very small compared to the original radius.}. This kind of collapse is called homologous collapse and the free-fall time when all gas reaches the ``singular point'' is given by:
$t_{ff} = \left(\frac{3\pi}{32G\rho_0}\right)^{1/2}$ where all variables are given previously in this section.
When temperatures increase, some of the heavier elements will ionize and the free electrons bonds with hydrogen. The $H^-$ ions increase the opacity drastically trapping the heat from gravitational potential energy more efficiently.
When this happens the collapse will be adiabatic instead of isothermal, and temperatures will increase.
When the gas becomes dominantly adiabatic, but still has no stable fusion process in the core the cloud of gas is called a protostar.
Small fusion processes and increased opcity increases the effective surface temperature and luminosity of the gas cloud.

Any inhomogeneities in density, and pressure gradient can cause fracturing of the collapsing gas, as can the presence rotation, turbulence, and magnetic fields. Meaning the subregion is divided into smaller regions where the density might not be big enough or several protostars can be created.
This might also lead to binary systems or systems with several stars.
The entire giant molecular cloud can also be considered a collapsing gas, but fracturing causes several stars to be born as separate entities inside.

When the density and temperature in the core becomes sufficiently high, the protostar will synthesize hydrogen into helium through the pp-chain, or if the star is massive enough the CNO-cycle.
This period of the stars life is the longest and is called the main sequence.
In the Hertzsprung Russel diagram this extended curve is well documented, and higher mass stars will find themselves higher in the diagram (more mass means more pressure which means more efficient fusion processes).
Throughout the stars life in the main sequence the luminosity and effective temperature will increase steadily as the overall mean molecular weight cahnges in the entire star.
The location in the Hertzsprung Russel diagram (i.e. the luminosity and effective surface temperature of the star) when the star first starts to burn steadily (when the star is born so to speak) is called the zero-age main sequence.

The time it takes for a cloud of gas to collapse and reach the zero-age main sequence is inversely proportional to it's mass, while the duration a star spends on the main sequence is roughly proportional to the inverse cube of the mass\comment{(from AST1100)}.
In short massive stars are quickly born and die more quickly, while smaller stars take alot more time. This makes the smaller stars more susceptible to effects from nearby massive stars that ionize or explode while the smaller stars are still forming.
Dispite this, observations show that the mass distribution function of stars massively favor low mass stars.

\comment{(include plot of initial mass distributions here?)}

During the main sequence, where hydrogen is burned into helium in the core, higher mass stars will have a much higher central temperature and density. This means that the hydrogen burning core will be dominated by the CNO-cycle, and the star will have a convective layer that develops in the envelope and stretches deep into the star.
Stars with lower mass on the other hand will have cores dominated by the pp-chain because their central temperature is lower. The energy transport will also be mostly radiative from the core out to the envelope.
Really low mass stars on the other hand will develop convective layer from the center outwards.
As the stars age, more hydrogen will burn into helium and the mean molecular weight will increase, steadily increasing the temperature, radius and luminosity of the stars on the main-sequence.

When the core of a low-mass star is depleted of hydrogen and filled with helium the pp-chain will stop in the core, but it will continue in a shell around the core due to high temperatures. The hydrogen burning shell around the core will provide more energy and cause the envelope to expand, this causes the luminosity to increase, but surface temperature to decrease.
higher mass stars will contract, and the convective layer disappears steadily as the core runs out of hydrogen fuel. The contraction heats the core and hydrogen shell burning will power the star.

As more helium is accreted onto the inert, isothermal helium core, which will collapse when it reaches the chandrasekhar limit.
The collapse of the core causes heating, which infaltes the envelope. Inflating the envelope causes the surface temperature to decrease, this is called the sub giant branch. The inflated envelope stabilizes and becomes convective from the large temperature gradient. The ffective energy transport causes the luminosity to increase into the red giant tip.
This leads to the first dredge-up where material from outside the core can be mixed into the upper envelope.
The collapsed core can now start fusing helium into carbon and oxygen through the triple alpha process. The core will then expand, cooling the hydrogen shell and decrease the overall luminosity of the star. Stars with lower masses will develop a electron degenerate core which will cause the core helium flash once the helium is ``ignited'' nearly simountaneously.

The envelope will contract following the expansion of the helium burning core, causing the effective surface temperature to rise. When stable radius, helium burning core and hydrogen burning shell is reached the star will have settled onto the horizontal branch. This is the main sequence equivalent of helium burning stars. As the helium is exhausted in the core, the core will start to contract, expanding the envelope, and the effective surface temperature will decrease toward the redder side of the horizontal branch.

When the heliumn has been expended in the core, leaving a inert core of carbon and oxygen with a helium burning shell around it.
The helium burning shell will dominate over the hydrogen burning shell lying on top of it, the increased temperature will cause the hydrogen burning shell to expand and essentially stop energy production.
When the helium burning shell exhausts all it's fuel the envelope will expand and become convective, the ensuing mix of material from bottom envelope (helium burning shell) to top envelope is called the second dredge-up. The convective energy transport is more effective, making the luminosity of the star increase. In the hertzsprung-russel diagram, this moves the star up into the asymptotic giant branch.
At this point, the hydrogen burning shell will dominate the energy production of the star once again. The ``ash'' from the hydrogen burning shell (the top shell) will ``rain'' down onto the inert helium burning shell (bottom shell). When the temperature is high enough and the bottom shell has enough material, the bottom shell of helium will ignite. Do to the isothermal layer of the helium shell, triple alpha burning will commence in the entire shell simountaneously, in an explosive fashion. This explosion, called the helium shell flash, is less explosive the the helium core flash, but might eject more material because it is closer to the surface.
When the helium has been exhausted once again, the shell compresses and the entire process repeats. The repetition of helium flashes is called the third dredge-up, mixing material from the hydrogen and helium shells into the upper envelope.

During the asymptotic giant branch stars loose alot of their material by ejection into the interstellar medium. This ejection can be from helium flashes, pulsations of the envelope, high luminosity, low surface gravity, high radiation pressure. The combination of effects is not surely determined, but simulations and observations show that the mass-loss must be great during this stage.

After the helium-flashes have subsided, the envelope has been ejected, the shell-burning have stopped, the star remains as a hot inert core of carbon and oxygen (with some hydrogen and helium surrounding it). This remnant is called a white dwarf.
This white dwarf will continue to glow until it has radiated away all it's thermal energy.

Evolution of massive stars:
Stars with mass $M \gtrsim 8 \msol$ evolve a bit differently. They have no helium flashes.
Their high mass means that the central density, pressure, and temperature will be higher.
The hydrogen burning core will fill up with helium, and when sufficient mass has been reached the helium will start to fuse into carbon and oxygen through the triple alpha process and hydrogen will burn in a shell around it.
The carbon in the core will then continue to fuse with more helium into oxygen and neon, with some sodium and magnesium produced.
The oxygen in the core will eventually start to fuse into silicon, and the silicon will eventually start to fuse into sulfur, argon and iron.
In this high temperature and high density environment, this process will not be straight forward, many different carbon isotopes will fuse with other particles into many different heavier isotopes. The details above outlines the general trend.
Assuming that there is an equilibrium of nuclear reactions the stars interior will resemble an onion like shell structure with the heaviest elements deepest in the star.

Fusion processes cannot produce excess energy for elements heavier then iron. Although trace amounts of heavier elements can be created from the excess thermal energy.
\iffalse
#photo disintegration free protons and electrons
Due to the high temperature and density along with the presence of heavy isotopes, can cause protons and electrons to be ejected from their isotopes.
\fi
In the centre of the core, the free electrons can merge with the free protons to create neutrons and release neutrinos
\comment{add nuclear reactions}
The sudden loss of electrons causes the electron degeneracy to drop suddenly and the centre of the core will collapse supersonically until the density is roughly 3 times the nucleon density. At this point the centre of the core, consisting of mostly neutrons will experience a repulsive effect of the strong nuclear force. This is equivalent to a pauli exclusion principle of neutrons.
The repulsive force causes the core to stiffen and rebound. The shock from the rebounding core meets the falling core on top, causing a shockwave that travels outward from the inner core.

Simulations suggest that the shockwave released will be absorbed by the surrounding layers. The stalled shockwave leaves a shell of high density. This shell is dense enough to absorb a significant amount of the neutrinoes released during collapse. If a small amount of the neutrino energy is transferred to the stalled shockwave it will restart and eject the surrounding layers into the interstellar medium.
The travelling shockwave can be observed as a type Ib, Ic, or II supernova, also called a core collapse supernova. The remnant of such an event will be a neutron star or (if the mass is great enough to overcome the repulsive force of the strong nuclear force) a black hole.

\subsection{Type 1a supernova}
``When a white dwarf (WD) composed of carbon and oxygen accreting mass from a companion star in
a binary system approaches the Chandrasekhar mass [$M_{Ch} \simeq 1.38$ solar masses (\msol)], high temperature
causes the ignition of explosive nuclear burning reactions that process stellar material and produce energy.
The star explodes leaving no remnant, producing a Type Ia supernova (SNIa) (K. Nomoto, F.-K. Thielemann, K. Yokoi, ApJ 286, 644 (1984)).''\mycite{mazzali07}

In the thermonuclear explosion iron peak elements(mostly Ni and Fe isotopes and below) are synthesized and ejected into the interstellar medium. During accretion, helium and hydrogen burmning layers develop and helium flashes occur. These flashes can cause major mixing of hydrogen into the carbon-layers which again can cause neutron producing reactions in greater numbers. Neutron capture processes can occur on the surface of type 1a supernovae if the produced neutron densities are high enough\mycite{nomoto84}. The isotope distributions also seemed to fill in some missing yields from type II supernovae.

Typical type 1a supernovae are heated from the deacy of \isotope{56}{28}{Ni} and will eject $\simeq 1.4 \msol$ of material at a ejecta velocity of $\simeq10 Mm s^{-1} \simeq 0.03c$ \comment{(cite tanaka 16)}

\subsection{Neutron star mergers}
The idea of mergers of compact objects(either neutron stars or black holes) by emission of gravitational waves have been around for a long time. The general concept is build on two compact objects orbiting eachother, interacting with the spatial curvature and creating ripples. These ripples maifest as waves in the fabric of space-time and carry gravitational energy away from two compact objects. The two objects move closer as a result of the lost energy, and increases the orbital velocity accordingly.
These gravitational waves distort space itself and can be detected by large laser interferometers that detect spatial disturbances smaller then the width of a nucleus \comment{(add reference to ligo paper)}.

Emission from a binary neutron star merger, or a kilonova, is heated from the decay of r-process elements \comment{add reference to nuclear physics section}. The ejecta mass and velocity are debated, but estimates are around $v=30-60 Mms^{-1} = 0.1-0.2c \quad m \simeq 0.01 \msol$\comment{add reference to Tanaka 2016 review article}.

During a merger between two neutron star mergers, the two stars move closer to eachother over time from gravitational radiation. When they are close enough to eachother they will disrupt eachothers surface, and surround the merging bodies in a cloud of neutron heavy material that is ejected into the interstellar medium. The force that pulls apart the neutron stars surface is only eachothers gravitational pull and centripetal force.
As the main bodies merge, a shock drives ejecting of material that will bombard the surrounding cloud.
As the cloud ejects from the colliding stars, the density of neutronmatter will drop until extremely neutron heavy nuclei will form, like droplets from steam. These nuclei are unstable and will \betadecay to more stable isobars. These heavy nuclei will act as seeds for the neutronrich shockwave emitting from the collision. The stream of very dense, high-velocity neutrons onto seed of heavy nuclei is the perfect recipe for r-process nucleosynthesis.
\comment{this is also from tanaka-article, do I cite again or move the citation}
