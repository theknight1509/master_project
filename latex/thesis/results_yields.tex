\section{Uncertainty of yields}

\subsection{quick recap of purpose and method}
%vary yields, keep everything else constant
The 'Omega' model with 'Eris' bestfit parameters is used to calculate the amount on \re{187}.
A single function of 'Omega' is overwritten in order to change the stellar yields of \re{187}.
The function multiplies the stellar yield of \re{187} ($Y_\re{187}$) with a constant, the new yield is denoted $\hat{Y}_\re{187}$.

The yield table for binary neutron star mergers is taken from \cite{arnould07} and is the calculated distribution of r-process isotopes in the \sos. The \sos-distribution of isotopes is measured from the solar photosphere and chondrite meteorites in \cite{landolt93}. The s-process distribution can be calculated from nuclear reaction networks, and the r-process/p-process distributions are calculated by subtracting the s-process distribution after fitting to the \sos-abundances.
The basic assumption is that all neutron star mergers eject material with the same isotopic distribution, and this distribution matches the observed solar distribution.
In short, the yield-table used is the \textit{observed} r-process distribution of the \sos, and this distribution is applied to every neutron star mergers.

In \eris-postprocessing\cite{shen15} the yield-table applied to all neutronstar mergers is a different one. The yield table was based on simulations by Rosswog\comment{(Need some reference?)} of neutron star mergers. Those yield-tables however does not contain any output of \re{187}, which is essential for the purpose of this thesis.

\textit{What are the underlying uncertainties of the yield-tables that we so generously apply to our simulations?}
The yield-tables represent the distribution of isotopes and elements ejected from a star during the end of it's lifetime. This information comes from stellar evolution codes like \cite{paxton11} and the nuclear reaction networks applied to them, like \cite{pignatari16}.
Uncertainties in nuclear reaction rates, stellar interior environments and ejecta composition all add together to create a total uncertainty of yield tables which are hard to separate.
In this case, the neutron star mergers are \textit{assumed} to be the dominant contributer to r-process isotopes in the galaxy, and the distribution is \textit{assumed} to follow observed r-process distribution in our \sos (this is ultimately an assumption of homogeneity).
Following these assumptions, the uncertainties in nucleosynthesis of r-process isotopes will simply be the uncertainty \sos-observations\cite{arnould07}.
Simulations suggest\comment{(some reference to Rosswog)} that the ejecta yield of binary neutron star mergers are somewhat dependant on the electron-fraction of the initial neutron stars, not taking inhomogeneities, rotation, and magnetic field of the initial neutron stars into account. Also given the rarity of kilonovas (compact object mergers) the assumption of homogeneity is questionable.

%write description and code to single-yield modification

\label{sec:mod-omega}

In order to manipulate the program \omegamodel\ without changing the source code, which can be found on \comment{website to omega}, inheritance of python-classes was used.

By creating a new python-class that inherits all methods from \omegamodel\ (which inherits many methods from \chemevol ), and making a new function with the same name as the function that sets the yield tables, the old funciton is overridden by the new.
The old function in \chemevol\ is called \verb|__set_yield_tables|, and the new function, \verb|chem_evol__set_yield_tables|, has all the same content and overrides the old function. By adding a few lines of code to the end of the yield-table function, a 'fudge factor' is multiplied to the yield of a single isotope across all yield-tables used.
The extra lines of code are shown in listings \ref{lst:mod-omega}.

\begin{lstlisting}[style=custompython, caption={\label{lst:mod-omega}Snippet of code added to the existing function \texttt{\_\_set\_yield\_tables} in \chemevol\ in \omegamodel-framework. The code-snippet multiplies the yield of isotope \texttt{self.experiment\_isotope} with a factor \texttt{self.experiment\_factor} for all yield-tables where the isotope can be found.}]
####################################################
### End of function as written in 'chem_evol.py' ###
""" 
Change ytables(multiply yields of 'isotope' with 'factor') with
value of self.experiment_factor to isotope corresponding to self.experiment_isotope
"""
####################################################

#AGB + massive stars, and pop3 stars
#loop over the different objects
for table_object, table_name in zip([self.ytables, self.ytables_pop3], ["agb/massive", "pop3"]):
    #get list of available metalicities
    loa_metallicities = table_object.metallicities
    for Z in loa_metallicities:
        #get list of masses for each metallicity
        loa_masses = table_object.get(Z=Z, quantity="masses")
        for M in loa_masses:
            #get current yield
            try:
                present_yield = table_object.get(M=M, Z=Z, quantity="Yields",specie=self.experiment_isotope)
            except IndexError: #isotope doesn't exist for this table
                continue
            #modify yield by some factor
            new_yield = present_yield*self.experiment_factor
            #"insert" new yield back into table
            table_object.set(M=M, Z=Z, specie=self.experiment_isotope, value=new_yield)

# SN1a, NS-NS merger, BH-NS merger
# get index of isotope
index_iso = self.history.isotopes.index(self.experiment_isotope)
#loop over different objects
for table_object, table_name in zip([self.ytables_1a, self.ytables_nsmerger, self.ytables_bhnsmerger], ["sn1a", "nsm", "bhnsm"]):
    #get list of available metalicities
    loa_metallicities = table_object.metallicities
    #loop over metallicities
    for i_Z, Z in enumerate(loa_metallicities):
        #get current yield
        try:
            present_yield = table_object.yields[i_Z][index_iso]
        except IndexError: #this means that isotope doesn't exist for this table
            continue
        #modify yield by some factor
        new_yield = present_yield*self.experiment_factor
        #"insert" new yield back into table
        table_object.yields[i_Z][index_iso] = new_yield
\end{lstlisting}


The isotopic uncertainties of the observed s-process and r-process in the \sos\ are taken from \cite{arnould07} and \cite{landolt93}
\importantcomment{Add table with uncertainties}



%\comment{add single uncertainty of %\re{187}, \re{185}, \eu{x}-isotopes, and \os{x}-isotopes} }
%Plots of Eu-151
\setlength{\subfigwidth}{0.5\textwidth}
\begin{figure}
  \begin{subfigure}{\subfigwidth}
    \centering
    \includegraphics[width=0.9\linewidth]{results/plots_yields/mdot_time_Eu-151.png}
    \caption{Time evolution of \eu{151} ejected from supernovae and neutron star mergers.}
  \end{subfigure}
  \begin{subfigure}{\subfigwidth}
    \centering
    \includegraphics[width=0.9\linewidth]{results/plots_yields/ism_time_Eu-151.png}
    \caption{Time evolution of the total mass of \eu{151} in the galaxy over time.}
  \end{subfigure}
  \begin{subfigure}{\subfigwidth}
    \centering
    \includegraphics[width=0.9\linewidth]{results/plots_yields/mdot_hist_Eu-151.png}
    \caption{Sum total of mass of \eu{151} ejected from supernovae and neutron star mergers.}
  \end{subfigure}
  \begin{subfigure}{\subfigwidth}
    \centering
    \includegraphics[width=0.9\linewidth]{results/plots_yields/ism_hist_Eu-151.png}
    \caption{Sum total of mass of \eu{151} in the interstellar medium of the galaxy}
  \end{subfigure}
  \caption{Resulting evolution of \eu{151}-mass in a chemical evolution galaxy model.}
\end{figure}
%table of statistics input uncertainty vs. output uncertainty
\begin{table}
  \input{results/data_yields/table_Eu-151.dat}
\end{table}

%Plots of Eu-153
\begin{figure}
  \begin{subfigure}{\subfigwidth}
    \centering
    \includegraphics[width=0.9\linewidth]{results/plots_yields/mdot_time_Eu-153.png}
    \caption{Time evolution of \eu{153} ejected from supernovae and neutron star mergers.}
  \end{subfigure}
  \begin{subfigure}{\subfigwidth}
    \centering
    \includegraphics[width=0.9\linewidth]{results/plots_yields/ism_time_Eu-153.png}
    \caption{Time evolution of the total mass of \eu{153} in the galaxy over time.}
  \end{subfigure}
  \begin{subfigure}{\subfigwidth}
    \centering
    \includegraphics[width=0.9\linewidth]{results/plots_yields/mdot_hist_Eu-153.png}
    \caption{Sum total of mass of \eu{153} ejected from supernovae and neutron star mergers.}
  \end{subfigure}
  \begin{subfigure}{\subfigwidth}
    \centering
    \includegraphics[width=0.9\linewidth]{results/plots_yields/ism_hist_Eu-153.png}
    \caption{Sum total of mass of \eu{153} in the interstellar medium of the galaxy}
  \end{subfigure}
  \caption{Resulting evolution of \eu{153}-mass in a chemical evolution galaxy model.}
\end{figure}
%table of statistics input uncertainty vs. output uncertainty
\begin{table}
\input{results/data_yields/table_Eu-153.dat}
\end{table}

%Plots of Os-189
\begin{figure}
  \begin{subfigure}{\subfigwidth}
    \centering
    \includegraphics[width=0.9\linewidth]{results/plots_yields/mdot_time_Os-189.png}
    \caption{Time evolution of \os{189} ejected from supernovae and neutron star mergers.}
  \end{subfigure}
  \begin{subfigure}{\subfigwidth}
    \centering
    \includegraphics[width=0.9\linewidth]{results/plots_yields/ism_time_Os-189.png}
    \caption{Time evolution of the total mass of \os{189} in the galaxy over time.}
  \end{subfigure}
  \begin{subfigure}{\subfigwidth}
    \centering
    \includegraphics[width=0.9\linewidth]{results/plots_yields/mdot_hist_Os-189.png}
    \caption{Sum total of mass of \os{189} ejected from supernovae and neutron star mergers.}
  \end{subfigure}
  \begin{subfigure}{\subfigwidth}
    \centering
    \includegraphics[width=0.9\linewidth]{results/plots_yields/ism_hist_Os-189.png}
    \caption{Sum total of mass of \os{189} in the interstellar medium of the galaxy}
  \end{subfigure}
  \caption{Resulting evolution of \os{189}-mass in a chemical evolution galaxy model.}
\end{figure}
%table of statistics input uncertainty vs. output uncertainty
\begin{table}
  \input{results/data_yields/table_Os-189.dat}
\end{table}

%Plots of Os-192
\begin{figure}
  \begin{subfigure}{\subfigwidth}
    \centering
    \includegraphics[width=0.9\linewidth]{results/plots_yields/mdot_time_Os-192.png}
    \caption{Time evolution of \os{192} ejected from supernovae and neutron star mergers.}
  \end{subfigure}
  \begin{subfigure}{\subfigwidth}
    \centering
    \includegraphics[width=0.9\linewidth]{results/plots_yields/ism_time_Os-192.png}
    \caption{Time evolution of the total mass of \os{192} in the galaxy over time.}
  \end{subfigure}
  \begin{subfigure}{\subfigwidth}
    \centering
    \includegraphics[width=0.9\linewidth]{results/plots_yields/mdot_hist_Os-192.png}
    \caption{Sum total of mass of \os{192} ejected from supernovae and neutron star mergers.}
  \end{subfigure}
  \begin{subfigure}{\subfigwidth}
    \centering
    \includegraphics[width=0.9\linewidth]{results/plots_yields/ism_hist_Os-192.png}
    \caption{Sum total of mass of \os{192} in the interstellar medium of the galaxy}
  \end{subfigure}
  \caption{Resulting evolution of \os{192}-mass in a chemical evolution galaxy model.}
\end{figure}
%table of statistics input uncertainty vs. output uncertainty
\begin{table}
  \input{results/data_yields/table_Os-192.dat}
\end{table}

%Plots of Os-190
\begin{figure}
  \begin{subfigure}{\subfigwidth}
    \centering
    \includegraphics[width=0.9\linewidth]{results/plots_yields/mdot_time_Os-190.png}
    \caption{Time evolution of \os{190} ejected from supernovae and neutron star mergers.}
  \end{subfigure}
  \begin{subfigure}{\subfigwidth}
    \centering
    \includegraphics[width=0.9\linewidth]{results/plots_yields/ism_time_Os-190.png}
    \caption{Time evolution of the total mass of \os{190} in the galaxy over time.}
  \end{subfigure}
  \begin{subfigure}{\subfigwidth}
    \centering
    \includegraphics[width=0.9\linewidth]{results/plots_yields/mdot_hist_Os-190.png}
    \caption{Sum total of mass of \os{190} ejected from supernovae and neutron star mergers.}
  \end{subfigure}
  \begin{subfigure}{\subfigwidth}
    \centering
    \includegraphics[width=0.9\linewidth]{results/plots_yields/ism_hist_Os-190.png}
    \caption{Sum total of mass of \os{190} in the interstellar medium of the galaxy}
  \end{subfigure}
  \caption{Resulting evolution of \os{190}-mass in a chemical evolution galaxy model.}
\end{figure}
%table of statistics input uncertainty vs. output uncertainty
\begin{table}
  \input{results/data_yields/table_Os-190.dat}
\end{table}

%Plots of Os-188
\begin{figure}
  \begin{subfigure}{\subfigwidth}
    \centering
    \includegraphics[width=0.9\linewidth]{results/plots_yields/mdot_time_Os-188.png}
    \caption{Time evolution of \os{188} ejected from supernovae and neutron star mergers.}
  \end{subfigure}
  \begin{subfigure}{\subfigwidth}
    \centering
    \includegraphics[width=0.9\linewidth]{results/plots_yields/ism_time_Os-188.png}
    \caption{Time evolution of the total mass of \os{188} in the galaxy over time.}
  \end{subfigure}
  \begin{subfigure}{\subfigwidth}
    \centering
    \includegraphics[width=0.9\linewidth]{results/plots_yields/mdot_hist_Os-188.png}
    \caption{Sum total of mass of \os{188} ejected from supernovae and neutron star mergers.}
  \end{subfigure}
  \begin{subfigure}{\subfigwidth}
    \centering
    \includegraphics[width=0.9\linewidth]{results/plots_yields/ism_hist_Os-188.png}
    \caption{Sum total of mass of \os{188} in the interstellar medium of the galaxy}
  \end{subfigure}
  \caption{Resulting evolution of \os{188}-mass in a chemical evolution galaxy model.}
\end{figure}
%table of statistics input uncertainty vs. output uncertainty
\begin{table}
  \input{results/data_yields/table_Os-188.dat}
\end{table}

%Plots of Re-185
\begin{figure}
  \begin{subfigure}{\subfigwidth}
    \centering
    \includegraphics[width=0.9\linewidth]{results/plots_yields/mdot_time_Re-185.png}
    \caption{Time evolution of \re{185} ejected from supernovae and neutron star mergers.}
  \end{subfigure}
  \begin{subfigure}{\subfigwidth}
    \centering
    \includegraphics[width=0.9\linewidth]{results/plots_yields/ism_time_Re-185.png}
    \caption{Time evolution of the total mass of \re{185} in the galaxy over time.}
  \end{subfigure}
  \begin{subfigure}{\subfigwidth}
    \centering
    \includegraphics[width=0.9\linewidth]{results/plots_yields/mdot_hist_Re-185.png}
    \caption{Sum total of mass of \re{185} ejected from supernovae and neutron star mergers.}
  \end{subfigure}
  \begin{subfigure}{\subfigwidth}
    \centering
    \includegraphics[width=0.9\linewidth]{results/plots_yields/ism_hist_Re-185.png}
    \caption{Sum total of mass of \re{185} in the interstellar medium of the galaxy}
  \end{subfigure}
  \caption{Resulting evolution of \re{185}-mass in a chemical evolution galaxy model.}
\end{figure}
%table of statistics input uncertainty vs. output uncertainty
\begin{table}
  \input{results/data_yields/table_Re-185.dat}
\end{table}

%Plots of Re-187
\begin{figure}
  \begin{subfigure}{\subfigwidth}
    \centering
    \includegraphics[width=0.9\linewidth]{results/plots_yields/mdot_time_Re-187.png}
    \caption{Time evolution of \re{187} ejected from supernovae and neutron star mergers.}
  \end{subfigure}
  \begin{subfigure}{\subfigwidth}
    \centering
    \includegraphics[width=0.9\linewidth]{results/plots_yields/ism_time_Re-187.png}
    \caption{Time evolution of the total mass of \re{187} in the galaxy over time.}
  \end{subfigure}
  \begin{subfigure}{\subfigwidth}
    \centering
    \includegraphics[width=0.9\linewidth]{results/plots_yields/mdot_hist_Re-187.png}
    \caption{Sum total of mass of \re{187} ejected from supernovae and neutron star mergers.}
  \end{subfigure}
  \begin{subfigure}{\subfigwidth}
    \centering
    \includegraphics[width=0.9\linewidth]{results/plots_yields/ism_hist_Re-187.png}
    \caption{Sum total of mass of \re{187} in the interstellar medium of the galaxy}
  \end{subfigure}
  \caption{Resulting evolution of \re{187}-mass in a chemical evolution galaxy model.}
\end{figure}
%table of statistics input uncertainty vs. output uncertainty
\begin{table}
  \centering
  \caption*{title}
  \input{results/data_yields/table_Re-187.dat}
  \caption{caption}
\end{table}

%conclusion - linear relationship between input/output variation
% - most uncertainties come from stellar evolution simulations
\FloatBarrier
