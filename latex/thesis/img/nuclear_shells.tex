%use \paperheight as fig-height
%\begin{figure}
  \centering
  \includegraphics[width=\linewidth]{img/nuclear_shells.png}
  \caption[Nuclear shells]{
    \label{img:nuclear-shells}
    Energy states of the nuclear orbitals/shells. This shows how the energy-states group together to form clusters of energy-states separeted by so-called magic numbers.\\
    The energy states are grouped together by their principle quantum number $n$, with their orbital splitting $l$ shown in the left column. As can be seen, each orbital term greater then zero (s=0, p=1, ...) are split into two sub-levels determined by their spin-orbit terms in the second column from the left. The third column represent the number of nucleons possible per level, and the far right column indicate magic numbers\mycite{basdevant2005}{ch.2.4}.\\
    Image credit: Bakken at English Wikipedia [CC BY-SA 3.0], from Wikimedia Commons
  }
%\end{figure}
