\section{Cosmology}

\subsection{Baryonic matter}
The known elementary particles are fermions and bosons, where the fermions are divided into leptons and quarks.
By combining quarks into groups of two and three via the strong force mesons and baryons are created\mycite{basdevant2005}.
Most known matter in the universe is made up of electrons, protons and neutrons. Since protons and neutrons make up most of the mass of these particles it is common to refer to standard model particles as the baryonic mass component of the universe.

\subsection{Dark matter}
The rotation curves of galaxies depends on the force of attraction (gravity), which depends on the (enclosed) mass within the galaxy.
However the kinematics of stars does not reproduce the observable mass, even after taking gas into account. This suggests a presence of particles that do not interact with regular baryonic matter. The phenomenon was dubbed ``dark matter'' and is believed to only interact with baryonic matter through gravity\mycitetwo{carroll2007}{ch.24.3}.

\subsection{Dark Energy}
Hubble calculated the distance, $d$, to 18 galaxies by means of Cepheid variable stars, and combining his results with the velocity, $v$ from Slipher a linear relation was found, $v=H_0d$, with some constant $H_0$, the hubble constant.
Galaxies moving away with greater velocities at greater distances, suggests the universe is expanding.
This is later supported by additional observations and theories\mycitetwo{carroll2007}{ch.27.2}.

In solutions to the tensor-field equations from general relativity, such an expansion comes natural if one considers a cosmological constant $\Lambda$. Such a component introduces an acceleration of the universe and is often called dark energy, or cosmological constant.

In an expanding universe, there are two sets of coordinates. Real coordinates which map the distances in real space between galaxies and comoving coordinates which follow the expansion of the universe.

\subsection{$\Lambda$ CDM}
The standard model of cosmology is the combination of the above components; baryonic matter, cold dark matter (CDM) and dark energy ($\Lambda$). The individual components have been well established from observations, e.g. Wilkinson Microwave Anisotropy Probe \mycite{wmap3}.

\subsection{Big bang nucleosynthesis}
Stars create heavier elements from lighter elements and produce energy as a result. Given the age of the universe and the stellar populations helium could not have been created in stars in the observed abundances from extra solar stars.
Starting with the big bang model of the universe, what elements would have been synthezised to create the nuclear abundances that would later become the first stars?
After inflation separates quantum fluctuations into particles, the universe was very dense and very hot. All matter (baryons, leptons, and dark matter) and energy tightly packed, interating and coupled. As the universe expands temperature and density drops accordingly.
After the hadrons form, nuclear matter can form.
Due to thermal equilibrium between neutrinos, electrons, and baryons the neutron-proton ratio is related by the boltzmann distribution. At the temperatures of weak interaction freeze-out, when this thermal equlibirum is no longer valid, the neutron proton ratio is two-to-five.
Since it takes some time for nucleosynthesis to take place and eventually form nuclear particles that are not instantly photodisintegrated.
During this time free neutrons decay to protons with a half-life of ten minutes, diminishing the final neutron proton ratio to one-to-seven at the time of nucleosynthesis. This means that there are two neutrons for every fourteen protons when nuclei can form.
Some basic math produces one $\alpha$-particle for every twelve free protons. The mass fraction of helium is therefore one fourth of the total nuclear mass budget in the universe, while hydrogen makes of three fourths of the total budget.
More detailed calculations of nucleosynthesis yield trace amount of \isotope{1}{3}{H}, \isotope{2}{3}{He}, \isotope{3}{7}{Li}, \isotope{4}{7}{Be}, but the dominant products are \isotope{1}{1}{H} with $\simeq75\%$ of the mass in the universe and \isotope{2}{4}{He} with $\simeq25\%$.
%\importantcomment{Find alpha-beta-gamma-48, Rosswog-99, BBFH, Lattimer&Schramm}
\comment{\\ include image of big bang model}

\FloatBarrier
