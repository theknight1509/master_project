\frame{Theory (10 min)
  \begin{itemize}
  \item basics of nuclear physics and reactions
  \item neutron capture processes
  \item stellar evolution and galactic enrichment
  \item Omega
  \item Eris
  \end{itemize}
}

\frame{\comment{Nuclear physics} \\ atom + chart of nuclides \\ shell-model \\ reaction rates + \betadecay \\ neutron-capture reactions \\ slow and rapid neutron capture \\ climbing in the chart of nuclides \\ Re-Os system + analytical model}
\frame{\comment{Stellar environments} \\ AGB + massive + SN2 \\ SN1a + BNSM + BHNSM \\ yield-tables}
\frame{\comment{Galaxies} \\ gravitational collapse of gas and dark matter \\ star formation from GMC \\ inflow from surrounding medium \\ outflow from supernovae}
\frame{\comment{\eris} \\ SPH \\ properties \\ postprocessing \\ sfr + total mass + [O/H] + [Fe/H] + [Eu/H]}
\frame{\comment{\omegamodel} \\ SFR + timestep -> stellar mass formed \\ stellar mass formed -> stellar population \\ stellar population + yield tables + delay-time -> isotopic yields recycled into ISM + remnant \\ remnants -> secondary events}

\begin{frame}
  \centering
  \vfill
  What is a cosmic clock?
  \vfill
  Why use \re{187}-\os{187}?
  \vfill
\end{frame}

\begin{frame}
  \frametitle{\re{187}-\os{187}}
  Halflife: \todo
  Different sources: slow and rapid neutron capture process
\end{frame}

\begin{frame}
  \frametitle{nucleosynthesis}
  \begin{itemize}
  \item Fusion of lighter elements (up to iron)
  \item Neutron capture processes
    \begin{description}
    \item[slow] \betadecays before succesive neutron capture
    \item[rapid] capture multiple neutrons before \betadecay
    \end{description}
  \end{itemize}
\end{frame}

\begin{frame}
  \todo[insert tikz-figure from clayton]
\end{frame}

\begin{frame}
  \frametitle{Analytical models of \re{187}-\os{187} cosmic clock}
  \todo[calculations and citations to appendix A]
\end{frame}

\begin{frame}
  \frametitle{Observed isotope fraction from meteorites and solar atmosphere}
  \todo[calculatations and citations from appendix A]
\end{frame}

%% \begin{frame}
%%   \frametitle{Formation of stellar systems in the Galaxy}
%%   \begin{itemize}
%%   \item Collapse of gas in giant molecular clouds
%%   \item Recycling of material from explosive events
%%   \item Inflow and outflow of gas from circumgalactic and extragalactic medium
%%   \end{itemize}
%% \end{frame}

\begin{frame}
  \frametitle{Chemical enrichment of galactic medium}
  \todo[insert tikz-figure of recycling]
\end{frame}

\begin{frame}
  \frametitle{Explosive events}
  \begin{itemize}
  \item Asymptotic giant branch stars (not really explosive)
  \item Core collapse supernovae
  \item Type 1a supernovae
  \item Neutron star mergers
  \end{itemize}
\end{frame}

\begin{frame}
  \frametitle{Eris simulation}
  \begin{minipage}{0.45\linewidth}
    \todo[insert eris-image]
  \end{minipage}
  \hfill
  \begin{minipage}{0.45\linewidth}
    \begin{itemize}
    \item Smoothed particle hydrodynamics simulation \mycite{guedes11}
    \item \todo[Add more simulation details? number of particles etc.]
    \item Postprocessing to add rapid neutron capture elements from neutron star mergers \mycite{shen15}
    \end{itemize}
  \end{minipage}
\end{frame}

\begin{frame}
  \frametitle{Omega semianalytical model \mycite{cote16a}}
  \begin{itemize}
  \item SFR + timestep $\rightarrow$ stellar mass formed
  \item stellar mass formed $\rightarrow$ stellar population
  \item stellar population + yield tables + delay-time $\rightarrow$ isotopic yields recycled into ISM + remnant
  \item remnants $\rightarrow$ secondary events
  \end{itemize}
\end{frame}

\begin{frame}
  \centering
  \textbf{
    \Large
    \mytitle
  }
\end{frame}
