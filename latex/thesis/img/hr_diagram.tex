%use \linewidth as fig-width
\begin{figure}
  \centering
  \includegraphics[width=\linewidth]{img/hr_diagram.png}
  \caption[Hertzsprung-Russel diagram]{\label{img:hr-diagram}
    ``The most famous diagram in astronomy is the Hertzsprung-Russell diagram. This diagram is a plot of luminosity (absolute magnitude) against the colour of the stars ranging from the high-temperature blue-white stars on the left side of the diagram to the low temperature red stars on the right side.

This diagram below is a plot of 22000 stars from the Hipparcos Catalogue together with 1000 low-luminosity stars (red and white dwarfs) from the Gliese Catalogue of Nearby Stars. The ordinary hydrogen-burning dwarf stars like the Sun are found in a band running from top-left to bottom-right called the Main Sequence. Giant stars form their own clump on the upper-right side of the diagram. Above them lie the much rarer bright giants and supergiants. At the lower-left is the band of white dwarfs - these are the dead cores of old stars which have no internal energy source and over billions of years slowly cool down towards the bottom-right of the diagram.''
    Image/description credit: Richard Powell [CC BY-SA 2.5] \href{http://www.atlasoftheuniverse.com/hr.html}{atlas of the universe}
  }
\end{figure}
