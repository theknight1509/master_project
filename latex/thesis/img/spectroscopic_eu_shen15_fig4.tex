%use \figwidth for width of figure

\begin{figure}
  \centering
  \includegraphics[width=\figwidth]{img/spectroscopic_eu_shen15_fig4.png}
  \caption[\mycitetwo{shen15}{fig.4}]{
    \label{img:spectroscopic-eu}
    Top two panels show spectroscopic Eu in star particles against the formation time of said star particles. Time is shown in log-scale.
    The blue lines show mean, 68\%, and 95\% intervals of the stars.
    The bottom two panels show spectroscopic Eu in star particles relative to the iron abundance in said star particles.
    Dashed line represent the solar value, while shaded regions are the 68\% and 95\% intervals of the stellar data.
    All data is retrieved from a representative subsample of star particles at the end of the simulation (redshift zero).

    The two left panels show the data from the \eris-simulation when there is no metallicity floor (a metallicity floor means that all particles have a minimum amount of metals at the beginning of the simulation, usually $10^{-4}\times$solar metallicity) and no mixing of metals between gas particles.
    On the two right panels, a metallicity floor and mixing of metals are implemented. This effect appears to lower the spread of metals in star particles at all times, and this result is consistent for iron and oxygen too\mycite{shen15}.

    Plots and figures are taken from \mycitetwo{shen15}{fig.4}
  }
\end{figure}
