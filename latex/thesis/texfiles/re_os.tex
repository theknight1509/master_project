%\begin{samepage}
\pagebreak
\section{The \re{187}-\os{187} chronometer}
\nopagebreak
\begin{figure}[h]
  %include tikz if not already done

%define functions for squares
\newcommand{\drawsquare}[3]{ %arguments: x0, y0, width/2
  \draw (#1-#3, #2-#3) -- (#1-#3, #2+#3)
  -- (#1+#3, #2+#3) -- (#1+#3, #2-#3)
  -- (#1-#3, #2-#3);
}
\newcommand\drawnuclide[4]{ %arguments: square-x0, square-y0, square-width/2, nuclide
  \drawsquare{#1}{#2}{#3};
  \draw (#1,#2) node {#4}
}
\newcommand\fillrectangle[3]{ %arguments: x0, y0, width/2
  \fill[color=lightgray,opacity=0.2, pattern=north west lines, pattern color=darkgray]
  (#1-#3, #2-#3) rectangle (#1+#3, #2+#3)
}

\newlength{\halfwidthnuclides}
\newlength{\distancenuclides}
\newlength{\offset}
\setlength{\halfwidthnuclides}{5mm}
\setlength{\distancenuclides}{3\halfwidthnuclides}
\setlength{\offset}{0.5\halfwidthnuclides}

\usetikzlibrary{patterns}

%\begin{figure}
\centering
\begin{tikzpicture}
  %test in the middle
  %\draw (0,0) node {${}^{A+1}_{Z}$X};
  %\tikzsquare{0}{0}{\halfwidthnuclides}
  %Os-row on top of test
  \iffalse %old version
  \draw (-\distancenuclides,\distancenuclides) node {${}^{186}_{76}$Os};
  \tikzsquare{-\distancenuclides}{\distancenuclides}{\halfwidthnuclides}
  \draw (0,\distancenuclides) node {${}^{187}_{76}$Os};
  \tikzsquare{0}{\distancenuclides}{\halfwidthnuclides}
  \draw (\distancenuclides,\distancenuclides) node {${}^{188}_{76}$Os};
  \tikzsquare{\distancenuclides}{\distancenuclides}{\halfwidthnuclides}
  %stable Re-isotopes
  \draw (-\distancenuclides,0) node {${}^{185}_{75}$Re};
  \tikzsquare{-\distancenuclides}{0}{\halfwidthnuclides}
  \draw (\distancenuclides,0) node {${}^{187}_{75}$Re};
  \tikzsquare{\distancenuclides}{0}{\halfwidthnuclides}
  %W island of stability
  \draw (\distancenuclides,-\distancenuclides) node {${}^{186}_{74}$W};
  \tikzsquare{\distancenuclides}{-\distancenuclides}{\halfwidthnuclides}
  %shaded region of stability
  \fill[color=lightgray,opacity=0.1, pattern=north west lines, pattern color=darkgray]
  (-0.5\distancenuclides,-0.5\distancenuclides)
  rectangle (-1.5\distancenuclides,1.5\distancenuclides)
  rectangle (1.5\distancenuclides,0.5\distancenuclides)
  rectangle (0.5\distancenuclides,-1.5\distancenuclides);
  \fi

  %draw stable nuclei from clayton64 fig.1.
  %row1 - Os-184, blank, Os-186, Os-187, Os-188, Os-189
  \drawnuclide{-3\distancenuclides}{\distancenuclides}{\halfwidthnuclides}{\os{184}};
  \drawsquare{-2\distancenuclides}{\distancenuclides}{\halfwidthnuclides};
  \drawnuclide{-\distancenuclides}{\distancenuclides}{\halfwidthnuclides}{\os{186}};
  \drawnuclide{0}{\distancenuclides}{\halfwidthnuclides}{\os{187}};
  \drawnuclide{\distancenuclides}{\distancenuclides}{\halfwidthnuclides}{\os{188}};
  \drawnuclide{2\distancenuclides}{\distancenuclides}{\halfwidthnuclides}{\os{189}};
  %row2 - blank, blank, Re-185, blank, Re-187, blank
  \drawsquare{-3\distancenuclides}{0}{\halfwidthnuclides};
  \drawsquare{-2\distancenuclides}{0}{\halfwidthnuclides};
  \drawnuclide{-\distancenuclides}{0}{\halfwidthnuclides}{\re{185}};
  \drawsquare{0}{0}{\halfwidthnuclides};
  \drawnuclide{\distancenuclides}{0}{\halfwidthnuclides}{\re{187}};
  \drawsquare{2\distancenuclides}{0}{\halfwidthnuclides};
  %row3 - W-182, W-183, W-184, blank, W-186, blank
  \drawnuclide{-3\distancenuclides}{-\distancenuclides}{\halfwidthnuclides}{\w{182}};
  \drawnuclide{-2\distancenuclides}{-\distancenuclides}{\halfwidthnuclides}{\w{183}};
  \drawnuclide{-\distancenuclides}{-\distancenuclides}{\halfwidthnuclides}{\w{184}};
  \drawsquare{0}{-\distancenuclides}{\halfwidthnuclides};
  \drawnuclide{\distancenuclides}{-\distancenuclides}{\halfwidthnuclides}{\w{186}};
  \drawsquare{2\distancenuclides}{-\distancenuclides}{\halfwidthnuclides};
  %row4 - Ta-181, blank, blank, blank, blank, blank
  \drawnuclide{-3\distancenuclides}{-2\distancenuclides}{\halfwidthnuclides}{${}^{181}_{73}$Ti};
  \drawsquare{-2\distancenuclides}{-2\distancenuclides}{\halfwidthnuclides};
  \drawsquare{-\distancenuclides}{-2\distancenuclides}{\halfwidthnuclides};
  \drawsquare{0}{-2\distancenuclides}{\halfwidthnuclides};
  \drawsquare{\distancenuclides}{-2\distancenuclides}{\halfwidthnuclides};
  \drawsquare{2\distancenuclides}{-2\distancenuclides}{\halfwidthnuclides};

  %shaded region of stability
  \fillrectangle{-3\distancenuclides}{-2\distancenuclides}{0.5\distancenuclides}; %Ta-181
  \fillrectangle{-3\distancenuclides}{-\distancenuclides}{0.5\distancenuclides}; %W-182
  \fillrectangle{-2\distancenuclides}{-\distancenuclides}{0.5\distancenuclides}; %W-183
  \fillrectangle{-\distancenuclides}{-\distancenuclides}{0.5\distancenuclides}; %W-184
  \fillrectangle{-\distancenuclides}{0}{0.5\distancenuclides}; %Re-185
  \fillrectangle{-\distancenuclides}{\distancenuclides}{0.5\distancenuclides}; %Os-186
  \fillrectangle{0}{\distancenuclides}{0.5\distancenuclides}; %Os-187
  \fillrectangle{\distancenuclides}{\distancenuclides}{0.5\distancenuclides}; %Os-188
  \fillrectangle{2\distancenuclides}{\distancenuclides}{0.5\distancenuclides}; %Os-189
  %\fillrectangle{\distancenuclides}{0}{0.5\distancenuclides}; %Re-187
  %\fillrectangle{\distancenuclides}{-\distancenuclides}{0.5\distancenuclides}; %W-186

  %draw s-process path
  \draw [ultra thick, ->, blue] (-2\distancenuclides-\offset,-3\distancenuclides-\offset)
  -- (-3\distancenuclides+\offset,-2\distancenuclides-\offset) -- (-2\distancenuclides-\offset,-2\distancenuclides-\offset)
  -- (-3\distancenuclides+\offset,-\distancenuclides-\offset) -- (0-\offset,-\distancenuclides-\offset)
  -- (-\distancenuclides+\offset,0-\offset) -- (0-\offset,0-\offset)
  -- (-\distancenuclides+\offset,\distancenuclides-\offset) -- (3\distancenuclides-\offset,\distancenuclides-\offset);
  %draw s-process branching point
  \draw [ultra thick, dashed, blue] (-\distancenuclides+\offset,-\distancenuclides-\offset)
  -- (2\distancenuclides-\offset,-\distancenuclides-\offset) -- (\distancenuclides+\offset,0-\offset)
  -- (2\distancenuclides-\offset,0-\offset) -- (\distancenuclides+\offset,\distancenuclides-\offset);
  \draw [ultra thick, dashed, blue] (-\distancenuclides+\offset,0-\offset)
  -- (2\distancenuclides-\offset,0-\offset);

  %draw r-process paths
  \draw [ultra thick, dotted, ->, red] (0+\offset,-3\distancenuclides-\offset)
  -- (-2\distancenuclides+\offset,-\distancenuclides-\offset);
  \draw [ultra thick, dotted, ->, red] (\distancenuclides+\offset,-3\distancenuclides-\offset)
  -- (-\distancenuclides+\offset,-\distancenuclides-\offset);
  \draw [ultra thick, dotted, ->, red] (1.5\distancenuclides+\offset,-2.5\distancenuclides-\offset)
  -- (-\distancenuclides+\offset,0-\offset);
  \draw [ultra thick, dotted, ->, red] (3\distancenuclides+\offset,-3\distancenuclides-\offset)
  -- (\distancenuclides+\offset,-\distancenuclides-\offset);
  \draw [ultra thick, dotted, ->, red] (3\distancenuclides+\offset,-2\distancenuclides-\offset)
  -- (\distancenuclides+\offset,0-\offset);
  \draw [ultra thick, dotted, ->, red] (3\distancenuclides+\offset,-\distancenuclides-\offset)
  -- (\distancenuclides+\offset,\distancenuclides-\offset);

  %add text
  \addtolength\offset{1.8\offset}
  \draw (-3\distancenuclides, -3\distancenuclides) node[thick, draw, blue] {s-process};
  \draw (2\distancenuclides, -3\distancenuclides) node[thick, draw, red] {r-process};
  \draw (0,0+\offset) node[draw=magenta, magenta, rounded corners] {${\scriptscriptstyle T_{\beta}=89hr}$}; %halflife of Re-186
  \draw (0,-\distancenuclides+\offset) node[draw=magenta, magenta, rounded corners] {${\scriptscriptstyle T_{\beta}=76d}$}; %halflife of W-185
  \draw (\distancenuclides,0+\offset) node[draw=magenta, magenta, rounded corners] {${\scriptscriptstyle T_{\beta}=41.6Gyr}$}; %halflife of Re-187
  \addtolength\offset{-1.8\offset}
\end{tikzpicture}
\caption[Chart of nuclides from \mycitetwo{clayton64}{fig.1} around A=187]{\label{tikz:nuclide-chart}
  Chart of nuclides around massnumber 187, adopted from \mycitetwo{clayton64}{fig.1}.
  The stable nuclei are denoted with their chemical symbols.
  The path of the s-process follows the valley of stability (shaded region), and is drawn as a blue solid line.
  Neutrons are absorbed during the s-process until and unstable isotope is reached, the unstable nuclide then \betadecay{s} to the higher isobars\footnote{the highest stable isobar means the nuclei with the same amount of nucleons and the highest amount of protons}.
  R-process nuclei are already very neutron-rich, and \betadecay{s} to the highest stable isobar.
  The path of the r-process is shown as red dotted lines.
  \w{185}, and also \re{186}, are potential branching points (), and can cause branched s-proces paths that are shown as blue dashed lines.
  The half-lifes of these potential branching point nuclei, as well as the half-life of \re{187}, are written in magenta over the nuclei.
}

\end{figure}
\FloatBarrier
%\end{samepage}

In this work, \textit{cosmochronology} refers to dating astrophysical events based on nucleosynthesis.
A \textit{chronometer} is a pair of radioactive nuclei, with an appropriate halflife which allows us to do exactly that.
As mentioned in section \ref{sec:betadecay}, \re{187} decays to \os{187} with a half-life of 41.6 Gyr (\mycite{snelling15}).
This halflife is appropriate for dating Galactic ages, which occur on the scale of universal ages (the age of the universe is measured to 13.8 Gyrs \mycite{planck15}).
Because of this, \mycite{clayton64} suggest the \re{187}-\os{187} nuclei as Galactic chronometer, also called a cosmic clock.

From section \ref{sec:rncp} it is clear that synthesis of heavy nuclei can be divided into two processes, rapid and slow neutron capture (ignoring proton capture).
Figure \ref{tikz:nuclide-chart} show the s-process path and the r-process path in the region around the \re{187}-\os{187}-pair.
\os{187} and \os{187} can only be reached by the s-process, because the r-process material will \betadecay to \re{187} and \w{186} where the halflife is comparable to the age of the universe. Simontaneously, neglecting the \w{184} branching points, the s-process cannot reach \re{187} (\mycite{clayton64}).

This means that \re{187} is only synthesized through the r-process and \os{187} is synthesized through the s-process from \os{186} and cosmoradiogenic \betadecay from \re{187}.
By assuming a rate of events producing r-process elements, starting from the age of nucleosynthesis, the amount of cosmoradiogenic \os{187} compared to \re{187} in the interstellar medium Galaxy is calculable.
This ratio is assumed to be representative of the interstellar medium near the \sos\ during its formation.
The same ratio can be found from meteorites today. These meteorites condensed from the dust and gas in the early epoch of the \sos and have remained pure since then until they impacted the Earth.

In his model for cosmochronology of Re-Os, \mycite{clayton64} assumes a exponential decay of the rate of r-process events $A(t) = e^{-\Lambda t}$, where an infinite decay constant ($\Lambda\rightarrow\infty$) leads to a ``sudden synthesis'' model and a zero decay constant ($\Lambda\rightarrow 0$) leads to a ``uniform synthesis'' model.
A detailed calculation following these models are presented in appendix \ref{sec:calc-cosmo-chronology}.
\comment{\\make figure with clayton model and meteorite abundances}

\iffalse
\comment{include chart of nuclide section}
%% \begin{figure}
%%   %include tikz if not already done

%define functions for squares
\newcommand{\drawsquare}[3]{ %arguments: x0, y0, width/2
  \draw (#1-#3, #2-#3) -- (#1-#3, #2+#3)
  -- (#1+#3, #2+#3) -- (#1+#3, #2-#3)
  -- (#1-#3, #2-#3);
}
\newcommand\drawnuclide[4]{ %arguments: square-x0, square-y0, square-width/2, nuclide
  \drawsquare{#1}{#2}{#3};
  \draw (#1,#2) node {#4}
}
\newcommand\fillrectangle[3]{ %arguments: x0, y0, width/2
  \fill[color=lightgray,opacity=0.2, pattern=north west lines, pattern color=darkgray]
  (#1-#3, #2-#3) rectangle (#1+#3, #2+#3)
}

\newlength{\halfwidthnuclides}
\newlength{\distancenuclides}
\newlength{\offset}
\setlength{\halfwidthnuclides}{5mm}
\setlength{\distancenuclides}{3\halfwidthnuclides}
\setlength{\offset}{0.5\halfwidthnuclides}

\usetikzlibrary{patterns}

%\begin{figure}
\centering
\begin{tikzpicture}
  %test in the middle
  %\draw (0,0) node {${}^{A+1}_{Z}$X};
  %\tikzsquare{0}{0}{\halfwidthnuclides}
  %Os-row on top of test
  \iffalse %old version
  \draw (-\distancenuclides,\distancenuclides) node {${}^{186}_{76}$Os};
  \tikzsquare{-\distancenuclides}{\distancenuclides}{\halfwidthnuclides}
  \draw (0,\distancenuclides) node {${}^{187}_{76}$Os};
  \tikzsquare{0}{\distancenuclides}{\halfwidthnuclides}
  \draw (\distancenuclides,\distancenuclides) node {${}^{188}_{76}$Os};
  \tikzsquare{\distancenuclides}{\distancenuclides}{\halfwidthnuclides}
  %stable Re-isotopes
  \draw (-\distancenuclides,0) node {${}^{185}_{75}$Re};
  \tikzsquare{-\distancenuclides}{0}{\halfwidthnuclides}
  \draw (\distancenuclides,0) node {${}^{187}_{75}$Re};
  \tikzsquare{\distancenuclides}{0}{\halfwidthnuclides}
  %W island of stability
  \draw (\distancenuclides,-\distancenuclides) node {${}^{186}_{74}$W};
  \tikzsquare{\distancenuclides}{-\distancenuclides}{\halfwidthnuclides}
  %shaded region of stability
  \fill[color=lightgray,opacity=0.1, pattern=north west lines, pattern color=darkgray]
  (-0.5\distancenuclides,-0.5\distancenuclides)
  rectangle (-1.5\distancenuclides,1.5\distancenuclides)
  rectangle (1.5\distancenuclides,0.5\distancenuclides)
  rectangle (0.5\distancenuclides,-1.5\distancenuclides);
  \fi

  %draw stable nuclei from clayton64 fig.1.
  %row1 - Os-184, blank, Os-186, Os-187, Os-188, Os-189
  \drawnuclide{-3\distancenuclides}{\distancenuclides}{\halfwidthnuclides}{\os{184}};
  \drawsquare{-2\distancenuclides}{\distancenuclides}{\halfwidthnuclides};
  \drawnuclide{-\distancenuclides}{\distancenuclides}{\halfwidthnuclides}{\os{186}};
  \drawnuclide{0}{\distancenuclides}{\halfwidthnuclides}{\os{187}};
  \drawnuclide{\distancenuclides}{\distancenuclides}{\halfwidthnuclides}{\os{188}};
  \drawnuclide{2\distancenuclides}{\distancenuclides}{\halfwidthnuclides}{\os{189}};
  %row2 - blank, blank, Re-185, blank, Re-187, blank
  \drawsquare{-3\distancenuclides}{0}{\halfwidthnuclides};
  \drawsquare{-2\distancenuclides}{0}{\halfwidthnuclides};
  \drawnuclide{-\distancenuclides}{0}{\halfwidthnuclides}{\re{185}};
  \drawsquare{0}{0}{\halfwidthnuclides};
  \drawnuclide{\distancenuclides}{0}{\halfwidthnuclides}{\re{187}};
  \drawsquare{2\distancenuclides}{0}{\halfwidthnuclides};
  %row3 - W-182, W-183, W-184, blank, W-186, blank
  \drawnuclide{-3\distancenuclides}{-\distancenuclides}{\halfwidthnuclides}{\w{182}};
  \drawnuclide{-2\distancenuclides}{-\distancenuclides}{\halfwidthnuclides}{\w{183}};
  \drawnuclide{-\distancenuclides}{-\distancenuclides}{\halfwidthnuclides}{\w{184}};
  \drawsquare{0}{-\distancenuclides}{\halfwidthnuclides};
  \drawnuclide{\distancenuclides}{-\distancenuclides}{\halfwidthnuclides}{\w{186}};
  \drawsquare{2\distancenuclides}{-\distancenuclides}{\halfwidthnuclides};
  %row4 - Ta-181, blank, blank, blank, blank, blank
  \drawnuclide{-3\distancenuclides}{-2\distancenuclides}{\halfwidthnuclides}{${}^{181}_{73}$Ti};
  \drawsquare{-2\distancenuclides}{-2\distancenuclides}{\halfwidthnuclides};
  \drawsquare{-\distancenuclides}{-2\distancenuclides}{\halfwidthnuclides};
  \drawsquare{0}{-2\distancenuclides}{\halfwidthnuclides};
  \drawsquare{\distancenuclides}{-2\distancenuclides}{\halfwidthnuclides};
  \drawsquare{2\distancenuclides}{-2\distancenuclides}{\halfwidthnuclides};

  %shaded region of stability
  \fillrectangle{-3\distancenuclides}{-2\distancenuclides}{0.5\distancenuclides}; %Ta-181
  \fillrectangle{-3\distancenuclides}{-\distancenuclides}{0.5\distancenuclides}; %W-182
  \fillrectangle{-2\distancenuclides}{-\distancenuclides}{0.5\distancenuclides}; %W-183
  \fillrectangle{-\distancenuclides}{-\distancenuclides}{0.5\distancenuclides}; %W-184
  \fillrectangle{-\distancenuclides}{0}{0.5\distancenuclides}; %Re-185
  \fillrectangle{-\distancenuclides}{\distancenuclides}{0.5\distancenuclides}; %Os-186
  \fillrectangle{0}{\distancenuclides}{0.5\distancenuclides}; %Os-187
  \fillrectangle{\distancenuclides}{\distancenuclides}{0.5\distancenuclides}; %Os-188
  \fillrectangle{2\distancenuclides}{\distancenuclides}{0.5\distancenuclides}; %Os-189
  %\fillrectangle{\distancenuclides}{0}{0.5\distancenuclides}; %Re-187
  %\fillrectangle{\distancenuclides}{-\distancenuclides}{0.5\distancenuclides}; %W-186

  %draw s-process path
  \draw [ultra thick, ->, blue] (-2\distancenuclides-\offset,-3\distancenuclides-\offset)
  -- (-3\distancenuclides+\offset,-2\distancenuclides-\offset) -- (-2\distancenuclides-\offset,-2\distancenuclides-\offset)
  -- (-3\distancenuclides+\offset,-\distancenuclides-\offset) -- (0-\offset,-\distancenuclides-\offset)
  -- (-\distancenuclides+\offset,0-\offset) -- (0-\offset,0-\offset)
  -- (-\distancenuclides+\offset,\distancenuclides-\offset) -- (3\distancenuclides-\offset,\distancenuclides-\offset);
  %draw s-process branching point
  \draw [ultra thick, dashed, blue] (-\distancenuclides+\offset,-\distancenuclides-\offset)
  -- (2\distancenuclides-\offset,-\distancenuclides-\offset) -- (\distancenuclides+\offset,0-\offset)
  -- (2\distancenuclides-\offset,0-\offset) -- (\distancenuclides+\offset,\distancenuclides-\offset);
  \draw [ultra thick, dashed, blue] (-\distancenuclides+\offset,0-\offset)
  -- (2\distancenuclides-\offset,0-\offset);

  %draw r-process paths
  \draw [ultra thick, dotted, ->, red] (0+\offset,-3\distancenuclides-\offset)
  -- (-2\distancenuclides+\offset,-\distancenuclides-\offset);
  \draw [ultra thick, dotted, ->, red] (\distancenuclides+\offset,-3\distancenuclides-\offset)
  -- (-\distancenuclides+\offset,-\distancenuclides-\offset);
  \draw [ultra thick, dotted, ->, red] (1.5\distancenuclides+\offset,-2.5\distancenuclides-\offset)
  -- (-\distancenuclides+\offset,0-\offset);
  \draw [ultra thick, dotted, ->, red] (3\distancenuclides+\offset,-3\distancenuclides-\offset)
  -- (\distancenuclides+\offset,-\distancenuclides-\offset);
  \draw [ultra thick, dotted, ->, red] (3\distancenuclides+\offset,-2\distancenuclides-\offset)
  -- (\distancenuclides+\offset,0-\offset);
  \draw [ultra thick, dotted, ->, red] (3\distancenuclides+\offset,-\distancenuclides-\offset)
  -- (\distancenuclides+\offset,\distancenuclides-\offset);

  %add text
  \addtolength\offset{1.8\offset}
  \draw (-3\distancenuclides, -3\distancenuclides) node[thick, draw, blue] {s-process};
  \draw (2\distancenuclides, -3\distancenuclides) node[thick, draw, red] {r-process};
  \draw (0,0+\offset) node[draw=magenta, magenta, rounded corners] {${\scriptscriptstyle T_{\beta}=89hr}$}; %halflife of Re-186
  \draw (0,-\distancenuclides+\offset) node[draw=magenta, magenta, rounded corners] {${\scriptscriptstyle T_{\beta}=76d}$}; %halflife of W-185
  \draw (\distancenuclides,0+\offset) node[draw=magenta, magenta, rounded corners] {${\scriptscriptstyle T_{\beta}=41.6Gyr}$}; %halflife of Re-187
  \addtolength\offset{-1.8\offset}
\end{tikzpicture}
\caption[Chart of nuclides from \mycitetwo{clayton64}{fig.1} around A=187]{\label{tikz:nuclide-chart}
  Chart of nuclides around massnumber 187, adopted from \mycitetwo{clayton64}{fig.1}.
  The stable nuclei are denoted with their chemical symbols.
  The path of the s-process follows the valley of stability (shaded region), and is drawn as a blue solid line.
  Neutrons are absorbed during the s-process until and unstable isotope is reached, the unstable nuclide then \betadecay{s} to the higher isobars\footnote{the highest stable isobar means the nuclei with the same amount of nucleons and the highest amount of protons}.
  R-process nuclei are already very neutron-rich, and \betadecay{s} to the highest stable isobar.
  The path of the r-process is shown as red dotted lines.
  \w{185}, and also \re{186}, are potential branching points (), and can cause branched s-proces paths that are shown as blue dashed lines.
  The half-lifes of these potential branching point nuclei, as well as the half-life of \re{187}, are written in magenta over the nuclei.
}

%% \end{figure}
From the chart on can see the usual path for slow neutron capture along the valley of stability.
This is the main contribution to \os{187}. In a standard s-process analysis, \w{185} and \re{186} are unstable, and will decay before they can capture neutrons. The s-process path will never synthesize \re{187}. However if the neutron capture rate is comparable to the \betadecay rates of those nuclides a branching point can occur. A branching point is a point where the synthesizing path of the s-process split due to competing nuclear reactions. In this case a significant fraction of the s-process path can go through \re{186} to \re{187} or through \w{185} to \w{186} (which is stable) and onwards to \re{187}. Apart from these effects \re{187} is shielded from s-process contribution.

The rapid neutron capture maintains very high neutron numbers until the neutron source ``shuts off'', at that point the isotopes \betadecay to the valley of stability. Given the long \halflife of \re{187} it can be considered stable, so the \os{187} isotope is shielded from r-process contribution because almost all \betadecay on the 187-isobar will stop at \re{187}.

Since \re{187} is radioactive with a halflife of \comment{add here} some \re{187} in the interstellar or stellar medium will decay to \os{187}. This amount is called the cosmoradiogenic \os{187}. The fraction between cosmoradiogenic \os{187} and current \re{187} is given by the exponential decay-function (assuming ofcourse that the nuclear decay rate is constant for all time, including stellar environments), meaning the time of nucleosynthesis can be calculated from the observed fraction of daughter-parent nuclei.

Clayton attempts to calculate the fraction of cosmoradiogenic osmium and the age of nucleosynthesis from these principles\mycite{clayton64}.
A brief summary follows:

The abundance of \os{186} is due to s-process only, and the abundance of \os{187} is due to s-process (from \os{186}) and cosmo radiogenic enrichment from \re{187} beta-decay.
The s-process contribution from \os{186} is given by the \os{186} abundance and the ratio between the cross-section of the isotopes. It is shown from nebular Samarium that the two s-only isotopes have nearly identical cross-section times abundance. Extrapolating this to other s-process isotopes, the result is:
(denoting abundance of isotope by their chemical name instead of N for simplicity)
\begin{equation}
  \bar{\sigma}_{\os{186}}{\os{186}} = \bar{\sigma}_{\os{187}}{\os{186}_S}
  \quad \rightarrow \quad
  {\os{186}_S} = \frac{\bar{\sigma}_{\os{186}}}{\bar{\sigma}_{\os{187}}}{\os{186}}
\end{equation}
$\bar{\sigma}$ are the neutron-capture cross-sections averaged over the appropriate thermal velocity distributions.
The cosmoradiogenic component of \os{187} is therefor the remaining part.
\begin{equation}
  \os{187}_c = \os{187} - \os{187}_s = \os{187} - \frac{\bar{\sigma}_{\os{186}}}{\bar{\sigma}_{\os{187}}}{\os{186}}
\end{equation}
Rewriting equation to relative units.
\begin{equation}
  \frac{\os{187}_c}{\re{187}} = \frac{\os{187} - \frac{\bar{\sigma}_{\os{186}}}{\bar{\sigma}_{\os{187}}}{\os{186}}}{\re{187}} =
  \frac{
    \left(\frac{\os{187}}{Os}\right) -
    \frac{\bar{\sigma}_{\os{186}}}{\bar{\sigma}_{\os{187}}}
    \left(\frac{\os{186}}{Os}\right)
  }{
    \left(\frac{\re{187}}{Re}\right)
  }
  \left(\frac{Os}{Re}\right)
\end{equation}
Origin of time, t=0, is set to \sos formation. Assuming that r-process events are supernovae and they began occuring at time T before \sos formation(t=0), and the frequency of events decreases exponentially as $f_0e^{\Lambda t}$, where $f_0$ is the initial supernovae freqency.
According to this model, and Clayton, the amount of cosmoradiogenic \os{187} is given by
\begin{equation}
  \frac{\os{187}_c}{\re{187}} = \left[
    \frac{\Lambda-\lambda}{\Lambda}
    e^{\lambda T}
    \frac{1-e^{-\Lambda T}}{1-e^{-(\Lambda-\lambda) T}}
    \right] - 1
\end{equation}
And the two special cases. \\
Sudden synthesis ($\Lambda\rightarrow\infty$)
\begin{equation}
  \frac{\os{187}_c}{\re{187}} = e^{\lambda t} - 1
\end{equation}
Uniform synthesis ($\Lambda\rightarrow 0$)
\begin{equation}
  \frac{\os{187}_c}{\re{187}} = \frac{\lambda T}{1-e^{-\lambda t}} - 1
\end{equation}

\comment{end of direct summary}

In short, by modelling r-process nucleosynthesis and s-process nucleosynthesis, the fraction of \re{187}-\os{187} in the \sos can predict the time of nucleosynthesis from non-cosmological methods.
\comment{\\ What do I mean about the age of nucleosynthesis?}
\fi
