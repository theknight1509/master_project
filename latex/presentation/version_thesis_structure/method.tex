\begin{frame}
\frametitle{Methods}
\begin{itemize}
\item Fitting \omegamodel\ to data from \eris
\item Manipulate yields in \omegamodel
\item Main experiments
\item Postprocessing
\end{itemize}
\end{frame}

\begin{frame}
  \frametitle{Fitting \omegamodel\ to data from \eris}
  \begin{minipage}{0.45\linewidth}
    \begin{itemize}
    \item Rough model
    \item ``$\chi^2$-by-eye''
    \item Star formation rate, stellar mass, total mass, [O/H], [Fe/H], [Eu/H]
    \end{itemize}
  \end{minipage}
  \hfill
  %% \begin{minipage}{0.45\linewidth}
  %%   \begin{itemize}
  %%   \item Direct insertion
  %%   \item Mass content
  %%   \item type 1a supernovae
  %%   \item Neutron star mergers
  %%   \item Size of timesteps
  %%   \end{itemize}
  %% \end{minipage}
\end{frame}

\setlength\subfigwidth{0.45\linewidth}
\begin{frame}
  \frametitle{Fitting \omegamodel\ to data from \eris}
  \begin{figure}
    \centering
    \begin{subfigure}{1.1\subfigwidth}
      \centering
      \includegraphics[width=\linewidth]{\curdir{eris_sfr.png}}
    \end{subfigure}
    \begin{subfigure}{0.9\subfigwidth}
      \centering
      \includegraphics[width=\linewidth]{\curdir{various_initial_mass_functions.png}}
    \end{subfigure}
  \end{figure}
\end{frame}
%\setlength\subfigwidth{0.3\linewidth}
\begin{frame}
  \frametitle{Fitting \omegamodel\ to data from \eris}
  \begin{figure}
    \centering
    \begin{subfigure}{\subfigwidth}
      \centering
      \includegraphics[width=\linewidth]{\curdir{set_sn1a_plot_sn1a_num2_oxy.png}}
    \end{subfigure} \hfill
    \begin{subfigure}{\subfigwidth}
      \centering
      \includegraphics[width=\linewidth]{\curdir{set_sn1a_plot_sn1a_num2_iron.png}}
    \end{subfigure} \\
    \begin{subfigure}{\subfigwidth}
      \centering
      \includegraphics[width=\linewidth]{\curdir{set_nsm_plot_final_spectro.png}}
    \end{subfigure}
  \end{figure}
\end{frame}
\begin{frame}
  \frametitle{Size of time steps}
  \begin{figure}
    \centering
    \includegraphics[width=\linewidth]{\curdir{resolution_difference_euro_logged.png}}
  \end{figure}
\end{frame}

\begin{frame}
  \frametitle{Manipulate yields in \omegamodel}
  \begin{table}[h]
    \centering
    \begin{tabular}{lrrrrr}
\hline
 isotope   &   standard &    min &    max &   $\sigma_{lower}$ &   $\sigma_{upper}$ \\
\hline
 Re-187    &     0.0318 & 0.027  & 0.0359 &            -0.1509 &             0.1289 \\
 Re-185    &     0.0151 & 0.011  & 0.0176 &            -0.2715 &             0.1656 \\
 Os-188    &     0.0707 & 0.0633 & 0.0781 &            -0.1047 &             0.1047 \\
 Os-189    &     0.103  & 0.0961 & 0.109  &            -0.067  &             0.0583 \\
 Os-190    &     0.152  & 0.137  & 0.168  &            -0.0987 &             0.1053 \\
 Os-192    &     0.273  & 0.252  & 0.289  &            -0.0769 &             0.0586 \\
 Eu-151    &     0.0452 & 0.0267 & 0.0482 &            -0.4093 &             0.0664 \\
 Eu-153    &     0.0495 & 0.046  & 0.0526 &            -0.0707 &             0.0626 \\
\hline
\end{tabular}}
    \caption{Values and uncertainties of r-process nuclei near \re{187} from (Arnould et al. 2007)}
    %% \caption[Values and uncertainties of observed r-process abundances in the \sos]{\label{tab:arnould-rncp-uncertainty}
    %%   Values and uncertainties of r-process nuclei near \re{187} from \mycite{arnould07}.
    %%   The relative uncertainty, $\sigma$-values, are calculated on the assumption that min/max are the one-sigma standard deviations in either direction.
    %% }
  \end{table}
\end{frame}

%% \begin{frame}
%%   \frametitle{Chemical evolution of \re{187}}
%%   \setlength{\subfigwidth}{0.45\textwidth}
%%   \begin{figure}[h]
%%     \begin{subfigure}{\subfigwidth}
%%       \centering
%%       \includegraphics[width=0.9\linewidth]{\thesisdir{results/plots_yields/mdot_time_Re-187.png}}
%%       %\caption{Time evolution of \re{187} ejected from supernovae and neutron star mergers.}
%%     \end{subfigure}
%%     \begin{subfigure}{\subfigwidth}
%%       \centering
%%       \includegraphics[width=0.9\linewidth]{\thesisdir{results/plots_yields/ism_time_Re-187.png}}
%%       %\caption{Time evolution of the total mass of \re{187} in the galaxy over time.}
%%     \end{subfigure}
%%     \begin{subfigure}{\subfigwidth}
%%       \centering
%%       \includegraphics[width=0.9\linewidth]{\thesisdir{results/plots_yields/mdot_hist_Re-187.png}}
%%       %\caption{Sum total of mass of \re{187} ejected from supernovae and neutron star mergers.}
%%     \end{subfigure}
%%     \begin{subfigure}{\subfigwidth}
%%       \centering
%%       \includegraphics[width=0.9\linewidth]{\thesisdir{results/plots_yields/ism_hist_Re-187.png}}
%%       %\caption{Sum total of mass of \re{187} in the interstellar medium of the galaxy}
%%     \end{subfigure}
%%     %\caption[effect of nucelar uncertainties]{Resulting evolution of \re{187}-mass in a chemical evolution galaxy model.}
%%   \end{figure}
%% \end{frame}

%% \begin{frame}
%%   \frametitle{Statistical deviation of \re{187}}
%%   \begin{table}
%%     \centering
%%     \input{\thesisdir{results/data_yields/table_Re-187.dat}}
%%     %% \caption[effect of nuclear uncertainties]{\label{tab:re187-yield-uncertainties}
%%     %%   Comparison of output mass-uncertainty with input yield-uncertainty.
%%     %%   $\sigma_{init}$ is the relative uncertainty from table \ref{tab:arnould-rncp-uncertainty} for zero, one, and two sigmas in either direction.
%%     %%   $\sigma_{ISM}$ is the relative deviation between the resulting mass of the interstellar medium, at current time (Z=0) and integrated for all time ($\Sigma$).
%%     %%   $\sigma_{\dot{m}}$ is the relative deviation between the ejected mass per timestep into the interstellar medium, at current time (Z=0) and integrated for all time ($\Sigma$).
%%     %% }
%%   \end{table}
%% \end{frame}

\begin{frame}
  \frametitle{Main experiments}
  \begin{itemize}
  \item Draw random ``fudge-factor'' from gaussian distribution
  \item 1500 individual calculations
  \item \textbf{Yields}
  \item \textbf{Yields+IMFslope}
  \item \textbf{Yields+IMFslope+NSM}
  \end{itemize}
\end{frame}


